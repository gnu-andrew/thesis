% Thesis: Global Synchronisation
% Author: Andrew Hughes

\chapter{Global Synchronisation}
\label{globsync}

\section{Introduction}
\label{timing}

In this chapter, we cover a number of process calculi that incorporate
abstract time.  The notion of `time' is generally associated with
concrete real values, in units such as minutes and seconds.  Real-time
process calculi, such as those described in
\cite{tccs,satoh:distrib,aceto:timing,satoh:phd,beaten:timing,lee:realtime,brics:lee},
attempt to model this.  Instead, this section focuses on calculi that
use \emph{clocks} to provide global synchronisation, as introduced in
\ref{ccslimit}.  While \emph{local synchronisation} occurs between two
processes, the global form synchronises any number of processes using
a clock signal.  For this, we don't need to measure time itself; we
just need an external reference point that can be used to co-ordinate
events.

Our central focus here is on CaSE (\ref{case}) which we extend to
create our own calculus in chapter \ref{nt}.  However, we first turn
our attention to its predecessor, TPL.

\section{Temporal Process Language (TPL)}
\label{tpl}

Hennessy's Temporal Process Language (TPL) \cite{hennessy:tpl} extends
the CCS language discussed in \ref{ccs} with a single clock, akin to a
hardware clock which emits a signal at an arbitrary point in time.
These signal emissions are controlled by a concept known as
\emph{maximal progress}, which allows each process to make as much
progress as possible before the clock ticks.  Formally, this means
that all silent actions ($\tau$s) are performed before a $\sigma$
action (which represents the clock signal) occurs.

This is of little use unless the actions of the processes can actually
depend on the behaviour of the clock.  The two are related via the
addition of a \emph{timeout} operator.  This takes the form

\begin{equation}
\tpltimeout{E}{F}
\end{equation}

\noindent where $E$ and $F$ are processes.  A process of this form can
either follow a transition from the process $E$, leaving only the
continuation of $E$, or perform a $\sigma$ transition to become $F$.
In short, $F$ acts if $E$ \emph{times out} on the clock, $\sigma$.
This is similar to non-deterministic choice, in that we lose either
$E$ or $F$ as the result of performing a transition.

Here, however, the choice is determined by the clock, and thus
effectively by the other processes, as it is their behaviour which
controls when the clock will tick.  If $E$ can perform a silent action
($\tau$), then the clock $\sigma$ will be prevented from ticking by
maximal progress, forcing one of $E$'s transitions to be taken.  For
example, $\tpltimeout{\tau.E^\prime}{F}$ has only one transition
$\derives{\tau}$ which leads to the state $E^\prime$.  If, however, we
replace $\tau.E^\prime$ with $\nil$ to give
$\tpltimeout{\nil}{F}$, we have two possibilities:

\begin{enumerate}
\item $\tpltimeout{\nil}{F} \derives{\sigma} \nil$
\item $\tpltimeout{\nil}{F} \derives{\sigma} F$
\end{enumerate}

\noindent as both $\nil$ and the timeout produce $\sigma$ transitions.

With these additions, the problem of defining a suitable compositional
broadcast agent, as mentioned in \ref{ccslimit}, can be solved.
Recall the second solution, which used recursion.  Now, with the
addition of an external entity (the clock) and a way of relating it to
the processes involved (timeouts), a base case may be provided via
recognition of the point when no more synchronisations may occur.
This can be added to the earlier recursive solution

\begin{equation}
\mathbf{\mu X.\timeout{\overline{o}.X}{\sigma}{0}} \pc o.P \pc o.Q \pc o.R
\end{equation}

\noindent by simply adding a timeout which stops the recursion.  This
works because the synchronisations of the input processes with the
output of the broadcast agent generate silent actions and thus invoke
maximal progress.  While there is a choice between a silent action
(due to the broadcasting agent synchronising with an input) and a
clock tick, the silent action always takes precedence and thus every
possible synchronisation occurs.  Once no more synchronisations are
possible, the clock is allowed to tick and the recursion stops.

\section{Extending TPL}
\label{tplext}

The extensions to TPL considered here focus on expanding the
scalability of the language.  As demonstrated above, TPL adequately
provides for situations where an arbitrary number of processes must
synchronize.  But what happens when a solution, like the one above, is
integrated into a larger system?  With only one clock, further
problems occur.  The use of the clock in one subsystem may conflict
with its use in another, and there is no clock available to
co-ordinate the subsystems themselves.

The Calculus for Synchrony and Asynchrony (CSA) \cite{csa} extends TPL
with the idea of multiple clocks, drawn from PMC\footnote{PMC also
  differs from TPL in its use of \emph{insistent} actions; all must be
  performed before a clock tick.}\cite{pmc}. However, while having
multiple clocks allows the use of differing patterns of
synchronisation, it increases the number of clock ticks present within
the system.  With five clocks, even the nil process has five possible
transitions (as clocks idle over nil).

CSA solves this to a limited extent by localising maximal progress to
a pre-defined scope for each clock.  A more elegant solution is
provided in the Calculus for Synchrony and Encapsulation (CaSE)
\cite{CaSE}, which introduces a clock hiding operator into the syntax.
The effect of this is the introduction of \emph{synchronous
  encapsulation}, as hidden clocks emit $\tau$ actions (as opposed to
ticks) outside the operator's scope.  This can be used, in conjunction
with restriction, to produce a hierarchy of components.  The actions
of these subsystems can be represented purely as silent actions, and,
when combined with the global form of maximal progress introduced by
TPL and retained in CaSE, integrated into the `synchronous cycle'
\cite{CaSE} of clocks at the level above.

\section{Calculus of Synchronous Encapsulation (CaSE)}
\label{case}

The syntax for CaSE, given in \cite{norton05alg}, is as follows:
\begin{equation}
  \begin{aligned}
    \expr, \exprb\ ::=\ &
    \nil  \;\,|\,\; 
    \Delta \;\,|\,\; 
    \Delta_{\sigma} \;\,|\,\; 
    \alpha . \expr  \;\,|\,\;
    \expr + \exprb \;\,|\,\; 
   \expr \mathrel{\!|\!} \exprb \mid
    \timeout{\expr}{\sigma}{\exprb} \;\,|\,\; \\
    & \stimeout{\expr}{\sigma}{\exprb} \;\,|\,\; 
    \mu X . \expr \;\,|\,\; 
    X \;\,|\,\; 
    \expr \setminus a \;\,|\,\; 
    \expr / \sigma
  \end{aligned}
\end{equation}
where $\expr$ and $\exprb$ define possible process terms. We assume a
countable set of actions, $\actions = \names \cup \conames \cup
\{\tau\}$, ranged over by $\alpha$, where the elements of $\names$ are
drawn from an infinite set of \emph{names}, and $\conames$ is the
corresponding set of \emph{co-names}, $\{\overline{a} \mid a \in
\names\}$. $\timers$ is a countably infinite set of \emph{clocks} over
which $\sigma$ ranges. $X$ ranges over a countably infinite set of
variables, which are used to bind process behaviour in recursive
process definitions. $\nil$, $\alpha . \expr$, $\expr + \exprb$,
$\expr \mathrel{\!|\!} \exprb$, $\mu X . \expr$, $X$ and $\expr
\setminus a$ retain their behaviour defined in CCS, but now exhibit
additional actions due to the presence of clocks.

There are now transitions for the $\nil$ process, as, while the
process has no explicit behaviour, it can idle over the ticks of the
clocks.  This also applies to actions in general:

\begin{equation}
a.0 \derives{\sigma} a.0
\end{equation}

\noindent assuming a clock context containing just the one clock,
$\sigma$. Similarly, non-deterministic choice and parallel composition
exist through time, so both sides can evolve due to a clock tick,
while the operator remains in place.  This gives the following
possible derivations for $a.0\;|\;b.0$ (where $b \ne \overline{a}$):

\begin{enumerate}
\item $a.0\ |\ b.0 \derives{a} 0\ |\ b.0$
\item $a.0\ |\ b.0 \derives{b} a.0\ |\ 0$
\item $a.0 |\ b.0 \derives{\sigma} a.0\ |\ b.0$
\end{enumerate}

\noindent with the same clock context as above.  The third derivation
is duplicated for each available clock that can tick over both sides
of the composition.  In cases where both sides may synchronize,
causing a $\tau$ transition, this takes precedence over the clock
transitions, due to \emph{maximal progress} (see \ref{timing}) and the
original set of derivations for parallel composition (see \ref{ccs})
are available instead.

The changes to non-deterministic choice are simpler, as the operator itself
does not generate silent actions.  So, if both sides allow the clock to tick,
then the following derivations will occur:

\begin{enumerate}
\item $a.0\ +\ b.0 \derives{a} 0$
\item $a.0\ +\ b.0 \derives{b} 0$
\item $a.0\ +\ b.0 \derives{\sigma} a.0\ +\ b.0$
\end{enumerate}

\noindent again with the single clock, $\sigma$, as the context.

\subsection{Timeouts}

Moving on to the new operators, CaSE, as presented in
\cite{norton05alg}, includes two variants of the timeout operator,
first seen in TPL.  Recall from \ref{timing} that the operator
essentially allows a decision to be made, based on the presence of a
clock tick.  In the general scenario,

\begin{equation}
\timeout{E}{\sigma}{F}
\end{equation}

\noindent $F$ will act if $E$ fails to, prior to a clock tick.  If $E$
can perform a $\tau$ action, then this will prevent the clock tick and
$E$ will evolve. Both operators in CaSE maintain this core behaviour,
which is central to the concept of global synchronisation explained
earlier.

The difference between the two operators in CaSE lies in their
behaviour with regard to other clocks.  With the fragile timeout,
$\timeout{E}{\sigma}{F}$, any possible transition on $E$ will cause the
removal of the timeout.  So, with $\timeout{a.0}{\sigma}{b.0}$ and a clock
context of $\sigma$ and $\rho$, the following derivations can occur:

\begin{enumerate}
\item $\timeout{a.0}{\sigma}{b.0} \derives{a} 0$
\item $\timeout{a.0}{\sigma}{b.0} \derives{\sigma} b.0$
\item $\timeout{a.0}{\sigma}{b.0} \derives{\rho} a.0$
\end{enumerate}

\noindent where both the $a$ and the $\rho$ transition leave only the
left-hand side of the timeout.

The stable timeout differs by continuing to exist through time until
some action occurs.  While it exhibits the same behaviour in response
to actions or the tick of the specified clock, the ticks of other
clocks only cause the left-hand side to evolve; the timeout itself is
retained.  Thus, $\stimeout{a.0}{\sigma}{b.0}$ gives a different set
of derivations:

\begin{enumerate}
\item $\stimeout{a.0}{\sigma}{b.0} \derives{a} 0$
\item $\stimeout{a.0}{\sigma}{b.0} \derives{\sigma} b.0$
\item $\stimeout{a.0}{\sigma}{b.0} \derives{\rho} \stimeout{a.0}{\sigma}{b.0}$
\end{enumerate}

\noindent where the $\rho$ transition no longer causes the dissolution
of the timeout.

\subsection{Clock Stopping and Insistency}
\label{clockcontrol}

The remaining operators further control the behaviour of the clocks.
$\Delta$ prevents all clocks from ticking, while $\Delta_{\sigma}$
prevents only the ticks of the specified clock, $\sigma$.  $\Delta$ is
similar to the CCS version of $\nil$, as it has no possible
transitions.  $\Delta_{\sigma}$ exhibits transitions for all other
clocks within the current context.  So, for a context containing both
$\sigma$ and $\rho$, $\Delta_{\sigma}$ has a single transition,

\begin{equation}
  \Delta_{\sigma} \derives{\rho} \Delta_{\sigma}
\end{equation}

\noindent which is replicated for any other clocks in the context,
which are not equal to $\sigma$.

The stopping of clocks is used to provide \emph{insistency}.  Normally,
a process $a.P$ has two possible derivations:

\begin{enumerate}
  \item $a.P \derives{a} P$
  \item $a.P \derives{\sigma} a.P$
\end{enumerate}

\noindent with a clock context containing only $\sigma$.  To ensure
that the first of these two derivations occurs, or, in other words, to
\emph{insist} that $a$ is performed before the next tick of the clock,
$\sigma$, $\Delta$ is used.  The semantics for an insistent prefix,
$\underline{\alpha}.P$, may be given as:

\begin{equation}
\seml \underline{\alpha}.P \semr \eqdef \alpha.P + \Delta 
\end{equation}

\noindent where the presence of $\Delta$ prevents a $\sigma$
transition from occurring on the right-hand side of the choice, and
thus for the choice as a whole (as both sides must move through time
simultaneously).  This leaves only one available action,
$\derives{a}$, as required.  Clearly, insistency relative only to one
particular clock may also be defined in a similar manner, using
$\Delta_{\sigma}$ instead.

\begin{equation}
\seml \underline{\alpha}_{\sigma}.P \semr \eqdef \alpha.P + \Delta_{\sigma} 
\end{equation}

While on the subject of derived syntax, it is also possible to define
a clock prefix, akin to the existing action prefix:

\begin{equation}
\seml \sigma.P \semr \eqdef \stimeout{\nil}{\sigma}{P}
\end{equation}

\noindent where the stable timeout ensures that the $\sigma.P$ will be
retained until $\sigma$ ticks, despite the ticks of other clocks.  As
the only transitions for $\nil$ are clock ticks, only a tick from
$\sigma$ will cause the process to evolve and become $P$.

The two notions of a clock prefix and insistency can then be combined
to give an insistent clock prefix:

\begin{equation}
\seml \underline{\sigma}.P \semr \eqdef \stimeout{\Delta}{\sigma}{P}
\end{equation}

\noindent which differs from a standard clock prefix by only ever
allowing the one transition, $\underline{\sigma}.P \derives{\sigma}
P$, whereas $\sigma.P$ allows an arbitrary number of transitions from
other clocks before this occurs.

\subsection{Encapsulation}

Clock hiding is used to provide scoping for the ticks of a
clock.  Take the following situation,

\begin{equation}
\label{clockhidingex}
  (P / \sigma)\;|\;Q
\end{equation}

\noindent where $/ \sigma$ hides the clock, $\sigma$, so that its
ticks may only be seen by $P$.  $Q$ instead sees a silent action each
time $\sigma$ ticks.  Such clock hiding is central to the
encapsulation of components present in CaSE.  When coupled with
restriction, components can be made to emit only silent actions from
the perspective of external processes.

An operational semantics for CaSE can be given in terms of a labelled
transition system, $(\procs, \actions \cup \timers, \rightarrow)$,
where $\procs$ is the set of CaSE expressions formed from the above
syntax, $\actions$ and $\timers$ are as defined above and
$\mathop{\rightarrow} \mathrel{\subseteq} \procs \times (\actions \cup
\timers) \times \procs$ is the transition relation defined in Table
\ref{tab:casesemantics}.  We use $E$ and $F$ to range over process
terms ($\procs$), $\alpha$ over the set of actions ($\actions$),
$\sigma$ and $\rho$ over the set of clocks ($\timers$), $a$ and $b$
over the set of names ($\names$) and $\gamma$ over $\names \cup
\timers$. In Table \ref{tab:casederivation}, we show how the semantics
for CaSE expand on those for CCS\footnote{We go directly from CCS to
  CaSE in the table, but as noted above, some concepts were
  introduced by intermediate calculi such as TPL.}, presented in Table
\ref{tab:ccssemantics}.

\begin{table}
  \caption{CaSE Semantics}
 \label{tab:casesemantics}
  \shrule
 \vspace{-2mm}
 \begin{center}
 \begin{tabular}{rlrl}
     \Rule{Idle}
     {-}
     {\nil \lderives{\sigma} \nil}
     {}
     &
     \hspace{5mm}
     \quad \Rule{Act}
     {-}
     {\alpha . E \derives{\alpha} E}
     {}
     \\[3ex]
     \Rule{Patient\quad}
     {-}
     {a.E \derives{\sigma} a.E}
     {}
     &
     \hspace{5mm}
     \Rule{Stall}
     {-}
     {\Delta_{\sigma} \derives{\rho} \Delta_{\sigma}}
     {\rho \ne \sigma}
     \\[3ex]
     \Rule{Sum1}
     {E \derives{\alpha} E^\prime}
     {E + F \derives{\alpha} E^\prime}
     {}
     &
     \hspace{5mm}
     \Rule{Sum2}
     {F \derives{\alpha} F^\prime}
     {E + F \derives{\alpha} F^\prime}
     {}
     \\[3ex]
     \Rule{Sum3}
     {E \derives{\sigma} E^\prime, F \derives{\sigma} F^\prime}
     {E + F \derives{\sigma} E^\prime + F^\prime}
     {}
     &
     \hspace{5mm}
     \Rule{Par1}
     {E \derives{\alpha} E^\prime}
     {E \;|\; F \derives{\alpha} E^\prime \;|\; F}
     {}
     \\[3ex]
     \Rule{Par2}
     {F \derives{\alpha} F^\prime}
     {E \;|\; F \derives{\alpha} E \;|\; F^\prime}
     {}
     &
     \hspace{5mm}
      \Rule{Par3}
      {E \derives{a} E^\prime,
        F \derives{\overline{a}} F^\prime}
      {E \;|\; F \derives{\tau} E^\prime \;|\; F^\prime}
      {}
     \\[3ex]
      \Rule{Par4}
      {E \derives{\sigma} E^\prime,
        F \derives{\sigma} F^\prime,
        E \;|\; F \nderives{\tau}}
      {E \;|\; F \derives{\sigma} E^\prime \;|\; F^\prime}
      {}
     &
     \hspace{5mm}
      \Rule{FTO1}
      {E \nderives{\tau}}
      {\timeout{E}{\sigma}{F} \derives{\sigma} F}
      {}
     \\[3ex]
      \Rule{FTO2}
      {E \derives{\gamma} E^\prime}
      {\timeout{E}{\sigma}{F} \derives{\gamma} E^\prime}
      {\gamma \ne \sigma}
     &
     \hspace{5mm}
      \Rule{STO1}
      {E \nderives{\tau}}
      {\stimeout{E}{\sigma}{F} \derives{\sigma} F}
      {}
     \\[3ex]
      \Rule{STO2}
      {E \derives{\alpha} E^\prime}
      {\stimeout{E}{\sigma}{F} \derives{\alpha} E^\prime}
      {}
     &
     \hspace{5mm}
      \Rule{STO3}
      {E \derives{\rho} E^\prime, E \nderives{\tau}}
      {\stimeout{E}{\sigma}{F} \derives{\rho} \stimeout{E^\prime}{\sigma}{F}}
      {\rho \ne \sigma}
     \\[3ex]
      \Rule{Rec}
      {E \derives{\gamma} E^\prime}
      {\mu X.E \derives{\gamma} E^\prime \{ \mu X.E / X\}}
      {}
      &
     \hspace{5mm}
      \Rule{Res}
      {E \derives{\gamma} E^\prime}
      {E \res{a} \derives{\gamma} E^\prime \res{a}}
      {\gamma \ne a}
     \\[3ex]
     \Rule{Hid1}
     {E \derives{\sigma} E^\prime}
     {E / \sigma \derives{\tau} E^\prime / \sigma}
     {}
     &
     \hspace{5mm}
     \Rule{Hid2}
     {E \derives{\gamma} E^\prime}
     {E / \sigma \derives{\gamma} E^\prime / \sigma}
     {\gamma \ne \sigma}
     \\
 \end{tabular}
  \end{center}
  \shrule
\end{table}

\begin{table}
 \caption{Derivation of CaSE from CCS}
 \label{tab:casederivation}
  \shrule \centering
  \begin{tabular}{r|c}
    Rule in CCS & Use in CaSE \\
    \hline
    \new & $Idle$; Allows time to run over $\nil$ \\
    $Act$ & As in CCS \\
    \new & $Patient$; Allows time to run over the prefix $\alpha.E$ \\
    \new & Prevents the clock $\sigma$ from ticking \\
    $Sum1$ & As in CCS \\
    $Sum2$ & As in CCS \\
    \new & $Sum3$; Allows time to run over a summation \\
    $Par1$ & As in CCS \\
    $Par2$ & As in CCS \\
    $Par3$ & As in CCS \\
    \new & $Par4$; Allows time to run over $|$ when not prevented by $\tau$ transitions \\
    \new & $FTO1$; Allows $\sigma$ to tick, leaving $F$, when there are no $\tau$ transitions \\
    \new & $FTO2$; Allows anything but $\sigma$ to precede, leaving $E'$ \\
    \new & $STO1$; Allows $\sigma$ to tick, leaving $F$, when there are no $\tau$ transitions \\
    \new & $STO2$; Allows $\alpha$ transitions to precede, leaving $E'$ \\
    \new & $STO3$; Allows other clocks ($\rho$) to tick over $E$ leaving the timeout intact \\
    $Rec$ & As in CCS, but generalised to $\gamma$ \\
    $Res$ & As in CCS, but generalised to $\gamma$ \\
    \new & $Hid1$; Converts the transitions of a hidden clock, $\sigma$, to $\tau$ transitions \\
    \new & $Hid2$; Allows anything but $\sigma$ to precede, retaining clock hiding
  \end{tabular}
  \shrule
\end{table}

\section{Conclusion}
\label{timelimit}

The main advantage of the timed calculi we have discussed here is that
they allow, via the introduction of \emph{global synchronisation}, the
construction of systems on a larger scale than those that could be
created purely with CCS.  With CaSE, components can be created which
consist of multiple processes and clocks.  These can then be
successfully integrated together to form new components.

Global synchronisation allows the problem of defining a compositional
broadcast agent, cited earlier in \ref{ccslimit}, to be solved, but
these timed calculi still retain the other problems with CCS we
mentioned there.  None of TPL, PMC, CSA or CaSE explicitly includes
data within the model.  This is not necessarily a disadvantage, as it
is possible to model data implicitly, via the use of silent actions.
Including data explicitly in the model complicates formal reasoning
and equivalence theories, so we will also adopt the implicit approach.

More importantly, these calculi all still retain a static structure.
The scope of restriction or clock hiding doesn't change as the
processes evolve.  This prevents these calculi from being used to
model mobile systems where these elements do change, although they
are perfectly suited to modelling static dataflow-oriented systems
such as those in \cite{WICSA} and \cite{cashews-sem}.

In contrast, the following chapter contains a discussion of calculi
which, while lacking the scalability of the timed languages just
illustrated, can model \emph{mobile systems}.

