\chapter{Future Work}
\label{futurework}

This chapter briefly discusses future work on the calculus, with regard
to both the theoretical aspects (sections \ref{futuretypes} and
\ref{futureequiv}) and its applications (section \ref{futureapp}).

\section{The Type System}
\label{futuretypes}

The type system presented in \ref{typesys} is still incomplete.
Specifically, as previously mentioned, it lacks rules for handling
process mobility and the association between localities and processes.

A possible formalisation of the latter is given in Table
\ref{tab:futuretype}.  The locality is a member of group $g$, while the
process is a member of group $g'$.  The two are unified by the
requirement that $g'$ be a member of the set of groups that can stay in
localities of group $g$, giving a final typing of $Loc(g')$.

\begin{table}
  \caption{Types: Linking Processes to Localities}
  \label{tab:futuretype}
  \shrule
 \begin{center}
\begin{tabular}{c}
     \Rule{Loc}
     {\Gamma \vdash m : Loc (g),
     \Gamma \vdash P : Proc (g'),
     \Gamma \vdash B : BProc,
     \Gamma \vdash g : G,
     g' \in \mathscr{S}(G)}
     {\Gamma \vdash \loc{m}{P}{B}{\vec{\sigma}} : Loc(g')}
     {}
\end{tabular}
  \end{center}
  \shrule
\end{table}

The rule also requires that $B$ is a bouncer process.  Absent from the
rule is the handling of the set of clocks.  Ideally, the restriction
should be something along the lines of $\{ \Gamma \vdash \sigma : Clock
\;|\; \sigma \in \vec{\sigma}\}$, stating that every element of the
vector, $\vec{\sigma}$, is a clock.  But it is presently unclear whether
this is an appropriate way of handling this requirement.

The task of handling process mobility is more tricky.  Objective
mobility is the issue, as the process which emits the mobility primitive
is not the process which moves and thus its type doesn't change either.
Instead, the process which does move is only recognisable by its
provision of a corresponding action.  This seems to imply that a more
complex way of handling actions is required.  Another issue is how
groups are to be allocated to processes and localities.  It may be
necessary to add a form of type annotation or a group binder (e.g. ($\nu
G$); see \ref{ambienttypes}), as in the type systems based on the
ambient calculus.

On a more general note, it would be interesting to consider further the
typing of both actions and clocks.  At present, they just have the
simplest of types, but this could be extended to provide more
interesting behaviour.  Similarly, bouncers only have rudimentary types
and may possibly be changed to include more detail.

\section{Equivalence}
\label{futureequiv}

The main element lacking in the current version of TNT is some notion of
equivalence.  An equivalence notion is necessary to be able to compare
processes, with the additional benefit of being able to reduce them to a
simpler form.

Equivalence falls under two areas.  The first is structural congruence
(\ref{structcong}), which is present in many distributed calculi such as
the $\pi$ calculus and the ambient calculus.  The second area is
bisimulation (\ref{bisimulation}) and specifically, the extension of the
temporal observation bisimulation congruence given for CaSE.

\subsection{Structural Congruence}
\label{structcong}

The semantics of TNT, specifically those in Table \ref{tab:casesubset}
in section \ref{tntsemantics}, may benefit from a notion of structural
congruence.  A structural congruence relationship allows the
rearrangement of the structure of a process by deeming the two to be
equivalent.  A clear example of where this would be advantageous with
the current semantics is the rules for summation given in Table
\ref{tab:summation}.  

\begin{table}
  \caption{Semantics: Summation}
  \label{tab:summation}
  \shrule
 \begin{center}
 \begin{tabular}{rc}
     \Rule{Sum1}
     {E \derives{\alpha} E^\prime}
     {E + F \derives{\alpha} E^\prime}
     {}
     &
     \Rule{Sum2}
     {F \derives{\alpha} F^\prime}
     {E + F \derives{\alpha} F^\prime}
     {}
     \\[3ex]
 \end{tabular}
  \end{center}
  \shrule
\end{table}

The only difference between the two rules is the side of the summation
on which the $\alpha$ transition occurs.  A structural congruence rule
along the lines of:

\begin{equation}
E + F \equiv E + F
\end{equation}

\noindent would remove the need for one of these rules and make clear
the commutative nature of the operator.  The same is true of the
parallel composition operator, which again has two rules, $Par1$ and
$Par2$.

Another common role for structural congruence is to remove unneeded
elements.  In many calculi, $\nil$ is removed in this way.  For example,
the ambient calculus has a structural congruence rule of the form

\begin{equation}
E \pc \nil \equiv E
\end{equation}

\noindent which removes the superfluous $\nil$.  However, as was
discussed in \ref{addingmob}, this is more problematic in CaSE and TNT,
where $\nil$ is not without behaviour (or, more specifically,
transitions).  Such a rule would however be very useful, as many of the
examples in the previous chapter run in to problems with superfluous
$\nil$ processes appearing.

Any structural congruence relationship defined needs to integrate with
the existing semantics to be useful.  Table \ref{tab:structcong} gives
an additional rule which shows how the structural congruence interacts
with transitions.  If a process $E'$ can perform any action, $\gamma$,
and become $F'$, while both $E'$ and $F'$ are structurally congruent to
$E$ and $F$ respectively, then the same transition may occur from $E$ to
$F$.

\begin{table}
  \caption{Semantics: Summation}
  \label{tab:structcong}
  \shrule
 \begin{center}
 \begin{tabular}{c}
     \Rule{SCong\ }
     {E \equiv E', E' \derives{\gamma} F', F' \equiv F}
     {E \derives{\gamma} F}
     {}
 \end{tabular}
  \end{center}
  \shrule
\end{table}

\subsection{Bisimulation}
\label{bisimulation}

Any bisimulation theory for TNT will be based on the labelled transition
system defined by the semantics.  In particular, the semantics share a
lot in common with those of CaSE, for which a form of bisimulation-based
equivalence (temporal observation congruence) already exists.  With
respect to the changes involved in TNT, the additional transitions
provided by the mobility primitives are all silent actions, which are
discarded by an observation-based congruence.  As a result, it is
unclear what changes will actually be necessary to provide the same
behavioural theory for TNT as is defined by this congruence for CaSE.

Mobility affects the topology of the process, and thus the difference
between two processes which exhibit mobility is much clearer via a
structural comparison.  Hence, structural congruence may end up being
more important with respect to the addition of mobility.

\section{Applications}
\label{futureapp}

The primary application of TNT will be biological modelling.  This is an
area where physical location is important and mobility is realised as
shifts in topological composition.  Nested structures are important,
being clearly present in, for example, cell membranes.  Within the
literature, models of a similar style have already been applied,
including Cardelli's brane calculi \cite{brane04} and the membrane
computing of P systems (see section \ref{psystems}).  This will also be
an interesting area to consider in terms of diversity, providing a
pleasant contrast with the more theoretical notions implicit in
designing the calculus itself.

One particular case study that is already being considered is that of
quorum sensing (see section \ref{quorumsensing}).  In brief, this
focuses on the reactive behaviour of bacteria in relation to the current
level of concentration of a particular gene.  The multi-party
synchronization implicit in TNT is likely to be useful in this area.

In modelling this, it is expected that additions will be made to the
calculus to allow the use of probabilistic derivation, akin to the use
of the stochastic $\pi$ calculus and the Gilliespie algorithm in similar
studies (see section \ref{bioapps}).  In particular, this will hopefully
allow simulations to be run and the behaviour of the model to be
analysed.

In the longer term, probably outside the scope of this thesis, there may
be other domains in which TNT will find applicability.  Pervasive
computing is thought to be one such area, and this may make more
effective use of the type system than biological modelling.