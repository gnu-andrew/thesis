\chapter{Contributions and Future Work}
\label{futurework}

This final chapter summarises our contributions to knowledge from
\ref{contributions}, before turning to the discussion of possible future
work in \ref{futureworksect}.

\section{Our Contributions}
\label{ourcontributions}

We have presented the following contributions to knowledge which we
believe to be novel:

\begin{enumerate}[\hspace{0.5cm}\bfseries {C}1.]
\item Nomadic Time, an algebraic process calculus with
  \emph{compositional global synchronisation}, \emph{mobility} and
  security provision via the notion of `\emph{bouncers}' (see chapter
  \ref{nt}).  This includes:
\begin{enumerate}[\bfseries {C1.}1]
\item The merging of clock hiding from the CaSE process calculus
  \cite{CaSE} with the notion of distribution, so that the CaSE term
  $E / \sigma$ becomes $\loc{m}{E}{B}{\sigma}$ in Nomadic Time.  In
  CaSE, clock hiding is used as a form of encapsulation; hiding a
  clock makes its ticks appear as silent actions to processes outside
  the scope of the clock hiding operator.  We can think of the
  processes encapsulated by the scope of the clock hiding operator as
  forming a component, and so it seems natural to name it.  The
  localised form of clock hiding present in Nomadic Time allows this
  through the use of named \emph{environs} (such as $m$ in the
  previous example).  Note that we use a set of clocks in Nomadic
  Time, so a CaSE term like $E / \sigma / \rho$ can be simplified to
  just $\loc{m}{E}{B}{\sigma, \rho}$.
\item The combining of this localised form of CaSE with structural
  mobility primitives from the ambient calculus \cite{amb} to give a
  new mobile calculus with the local and global synchronisation
  properties of CaSE.  This adds a new process form, $\ambop . \expr$,
  to the existing syntax of CaSE, where $\ambop$ is one of
  $\tntin{m}$, $\tntout{m}$ and $\tntopen{m}$, each in turn exhibiting
  the behaviour of $\ambin{m}$, $\ambout{m}$ and $\ambopen{m}$ from
  the ambient calculus.  We change the name of the operators in
  Nomadic Time to make them more distinct from our existing names and
  co-names, which aren't present in the ambient calculus.
\item The addition of a pair of process mobility primitives which
  allow direct process movement by synchronising on a name.  We add
  two new alternatives to $\ambop$, namely $\procin{a}{m}$ and
  $\procout{a}{m}$.  Just as a process, $a.P$, can synchronise with
  $\overline{a}.Q$ in CCS or CaSE, it can also synchronise with
  $\procin{a}{m}.Q$ or $\procout{a}{m}.Q$ in Nomadic Time.  If such a
  synchronisation happens, not only does $a.P$ becomes $P$, but it is
  also moved inside or outside of $m$.  This kind of direct process
  migration appears in few process calculi, and we believe this
  particular form, which leverages the existing synchronisation
  process, to be unique.
\item The introduction of `bouncers', which add a security mechanism
  to the calculus.  The range of $\ambop$ is again extended to include
  $\bin$, $\bout$ and $\bopen$, which correspond to $\sambin{m}$,
  $\sambout{m}$ and $\sambopen{m}$ respectively from the safe ambient
  calculus.  The difference between their use in the safe ambient
  calculus and their use in Nomadic Time is that these co-actions must
  come from a process which is attached to the top-right of an environ
  such as $B$ in $\loc{m}{E}{B}{\sigma}$.  This is why the operators
  themselves do not need to name an environ.
\item The creation of a set of structural congruence laws for Nomadic
  Time (Table \ref{tab:structcong}), allowing process terms to be
  simplified and the number of semantic rules to be reduced.  With
  these rules in place, we were able to drop the $Sum2$ and $Par2$
  rules taken from CaSE's semantics.  They are also useful in cases
  where, for example, a number of processes have evolved to $\nil$
  e.g. $\nil \pc \nil \pc \nil$ can be rewritten as $\nil$ using
  $StrIdent$.
\item The provision of structured operational semantics for Nomadic
  Time (Tables \ref{tab:core} and \ref{tab:mobsubset}).  These extend
  those from CaSE as demonstrated in Table \ref{tab:derivation},
  extending the notion of prioritisation and adding new rules to
  handle the introduction of environs and mobility primitives.  The
  main contributions in the semantics come from formulating
  appropriate structural operational semantics for the new mobility
  primitives, especially with regard to the three-way synchronisation
  present in the process mobility operators, and the generalisation of
  ideas from CaSE to incorporate mobility.  The latter includes the
  creation of a notion of \emph{high priority transitions}; this
  includes $\tau$ transitions and the transitions which occur when a
  mobility synchronisation (e.g. $\tntin{m}$ and $\bin{m}$) takes
  place.  So, although a number of the rules are derived from CaSE,
  they have had to be extensively re-examined to ensure that they work
  correctly in the context of the new mobility primitives.
\item The demonstration of the properties of \emph{prioritisation} and
  \emph{time determinacy} inherent in the new calculus.  This follows
  on from the semantics and explicitly shows how the high priority
  transitions defined there prevent any clock transitions from taking
  place.
\end{enumerate}
\item The realisation of the aforementioned calculus as a \emph{design
  framework}, DynamiTE, through the implementation of its constructs
  as programmatic elements in the Java programming language (see
  chapter \ref{dynamite}).  This allows the specification of system
  interactions to be shifted directly from the theoretical domain into
  an implementation backed by a formal methodology, with the intention
  of improving industrial adoption of concurrent techniques.  This includes:
\begin{enumerate}[\bfseries {C2.}1]
\item The creation of a translation schema (Table
  \ref{tab:translationschema1}), mapping process terms in Nomadic Time
  to Java objects.  This was largely a mechanical process of turning
  the syntax into objects, but there are a few areas that required
  more finesse.  In particular, it was necessary to create $\nil$,
  $\Omega$, $\Delta$ and the bouncer primitives as singletons in order
  to avoid there being multiple distinct instances of them.  Having
  only a single instance of each reduces memory usage and means that
  two separate references to them can be judged equivalent by
  comparing just the references and not the objects referred to.  The
  \texttt{Action} framework is also notable, particularly as it is the
  abstract notion of \texttt{Tau} which connects the
  intercommunication mechanism provided by DynamITE to the user's own
  code.
\item The implementation of the operational semantics as methods in
  the appropriate Java objects defined in the schema.  This was again
  a largely mechanical process, but did require significant testing to
  get right.  It also had the additional benefit of showing whether
  the semantics performed as expected in a practical situation.
\item The design and implementation of a plugin framework, allowing
  the use of different process calculi and different \emph{side
    effects} as the result of performing a transition.  As with any
  plugin framework, the full merits of this remain to be discovered,
  as other alternate implementations are developed and used with it.
  It does provide an important separation between the calculus, the
  framework and the side-effecting mechanism which will hopefully mean
  that others can use DynamiTE even if they do not wish to use Nomadic
  Time.  We believe DynamiTE to be novel in this respect, allowing
  other calculi to be used and also for existing implementations to be
  extended; so calculi that extend each other in the theoretical
  domain can also do so at the implementation level.  This is the case
  with CCS, CaSE and Nomadic Time, which, as noted earlier, share the
  same implementation of recursion (\texttt{Rec}).
\item The design and implementation of the \emph{evolver} framework,
  allowing the execution semantics to be both clearly denoted and
  interchangeable.  By providing just two implementations in DynamiTE
  -- the \texttt{Simulator}, which explores all transitions in a
  depth-first fashion, and the \texttt{RandomExecutor}, which follows
  one transition at random -- it is already possible to evaluate the
  communication semantics with the former, change a single line of
  code, and execute the process (including side effects) with the
  latter.
\end{enumerate}
\item The optional addition of a \emph{type system} to Nomadic Time in
  order to allow movement restriction to be based on the group
  membership of processes (see chapter \ref{tnt}); we refer to this
  extended version as Typed Nomadic Time (TNT).  This includes:
\begin{enumerate}[\bfseries {C3.}1]
\item The design of a group type which can be applied to a Nomadic
  Time process to restrict movement.  Although the idea of a type
  system based on groups is derived from existing work on type systems
  for ambient calculi (see \ref{ambienttypes}), our particular form of
  group, using the sets $\mathscr{R}$ (\emph{resides in}),
  $\mathscr{O}$ (\emph{opens}), $\mathscr{L}$ (\emph{leaves}) and
  $\mathscr{E}$ (\emph{enters}) is believed to be novel.
\item The provision of typing rules (Tables \ref{tab:basictypes},
  \ref{tab:totypes}, \ref{tab:botypes} and \ref{tab:mobtypes}) for the
  new typed form of Nomadic Time.  Again, we believe these to be
  novel.  In a number of cases, the rules are obvious, with the
  majority of work focusing on handling what happens when two groups
  converge (T-Sum, T-Par, T-FTO and T-STO) and the actual use of the
  groups in the mobility rules (T-EnvIn, T-EnvOut, T-Open, T-ProcIn
  and T-ProcOut).  We believe the rules to be useful in providing an
  additional layer of security, above and beyond the bouncer
  mechanism, when required.
\item Proofs of type safety for the type system.  These were largely
  mechanical but areas such as handling substitution required
  significant thought, and they did expose some holes in the type
  system which led to it being further revised and rectified.
\item The extension of DynamiTE to handle type systems in general, and
  specifically TNT.  This has many similarities with \cont{2.1} and
  \cont{2.2}, with the processes in the translation schema from
  \cont{2.1} now implementing \texttt{TypedProcess} and the typing
  rules being implemented in a similar manner to the way the
  operational semantics were in \cont{2.2}.  In addition, we had to
  decide how to best represent the group type, leading to the
  \texttt{ProcessType} interface and its subclasses, and how to signal
  a failure to type a process in an appropriate way, for which a
  specific exception type was used.
\end{enumerate}
\end{enumerate}

\section{Future Work}
\label{futureworksect}

In this final section, we look at a number of ideas with regard to
further developing the process calculus Nomadic Time
(\ref{future:nt}), the DynamiTE framework (\ref{future:dynamite}) and
the type system (\ref{future:types}).  We also touch on other possible
applications for Nomadic Time and its typed derivative, TNT
(\ref{future:apps}).

\subsection{Nomadic Time}
\label{future:nt}

\subsubsection{Separation of Syntax}

At present, the syntax represents bouncers as a form of process for
simplicity.  This has the advantage of being minimal, but also means
that there are number of possible constructs that we would prefer to
deem illegal (e.g. the use of bouncers with the timeout operators or
$\Omega$ being used as a `normal' process term).  In chapter
\ref{tnt}, we showed how the type system could be used to make such
terms invalid, but in hindsight it would better to have a more verbose
syntax which rules these out from the start.

\subsubsection{Equivalence}

The main element lacking in the current version of Nomadic Time is
some notion of equivalence.  This is necessary to be able to compare
processes, and brings the additional benefit of being able to reduce
them to a simpler form.  This would also ripple through to DynamiTE in
the implementation of an \texttt{equals} method for the Nomadic Time
implementations of \texttt{Process}, which currently only return true
where both objects are the same instance.

Any bisimulation theory for NT would be based on the labelled
transition system defined by the semantics.  In particular, the
semantics share a lot in common with those of CaSE, for which a form
of bisimulation-based equivalence (\emph{temporal observation
  congruence}) already exists.  Nomadic Time introduces a number of
new transitions, including the three \emph{mobility transitions}
($\tin$, $\tout$ and $\topen$) and the component transitions formed
from terms in $\ambop$ which pair up to generate $\tin$, $\tout$ and
$\topen$.  Thus, any equivalence theory would need to consider issues
such as whether $\tntin{m}$ is equivalent to $\tntin{n}$; do we
consider only that an `in' action is being performed or do we also take
into account the specific environ involved?

Mobility affects the topology of the process, and thus the difference
between two processes may not be fully realised simply by looking at
transitions.  For example, \linebreak
$\locv{m}{\locv{n}{\tntin{m}.\nil}{\Omega}{\emptyset}}{\Omega}{\emptyset}$
and $\locv{m}{\tntin{m}.\nil}{\Omega}{\emptyset}$ both produce a
$\tntin{m}$ transition along with one for each member of $\timers$.
Assuming $\timers$ is the same for both, the difference in topology is
not discernable from merely the set of possible transitions.

\subsubsection{Scoping of Environ Names}

Back in \ref{addingmob}, we noted that Nomadic Time does not yet
support the scoping of names.  In the ambient calculus, $(\nu n)E$ is
used to restrict the scope of the ambient name, $n$, to within $E$,
thus allowing only mobility operations which form part of $E$ to use
the name $n$.  Instead, Nomadic Time currently assumes that all
environ names are available globally.  While in most cases this isn't
a problem, as references to environs relate to siblings or parents, it
becomes an issue if a process moves location and now refers to an
environ it wouldn't have before.

For example, in

\begin{equation}
\locv{n}{\locv{m}{\tntout{n}.P \pc \tntin{k}.Q}{B_1}{\sigma}}{B_2}{\rho} \pc \locv{k}{R}{B_3}{\gamma}
\end{equation}

\noindent the environ $m$ may move outside $n$\footnote{Assuming
  appropriate bouncers are available.}, due to $\tntout{n}.P$, giving:

\begin{equation}
\locv{n}{\nil}{B_2}{\rho} \pc \locv{k}{R}{B_3}{\gamma} \pc \locv{m}{P \pc \tntin{k}.Q}{B_1}{\sigma}
\end{equation}

\noindent In such a situation, $\tntin{k}.Q$ may now cause $m$ to move
inside $k$, but, prior to the move, it couldn't as $k$ wasn't a
sibling of $m$.  It is, of course, possible that such behaviour was
intentional and such changes in context are, after all, the point of
having migration in the calculus to begin with.  However, the current
calculus does not give the user the option of preventing such
situations from occurring.  Adding a scoping operator, in a similar
manner to the ambient calculus, would allow the two references to $k$
in the above to remain distinct from one another and prevent the
$\tntin{k}$ from operating on the $k$ environ.

Note that the above assumes that the bouncer $B_3$ will allow $m$ to
enter.  Similarly, the type of the processes could also be used to
control access to $k$ with TNT.  Both these existing mechanisms reduce
the damage inherent in the use of global names, but they don't remove
it altogether.

\subsubsection{Bigraphs}

Back in \ref{bigraphs}, we look at Milner's unifying framework of
bigraphs \cite{bigraph1} and saw how these could be used to represent
both features of both the $\pi$ calculus and the ambient calculus.  It
would be interesting to see if Nomadic Time could also be represented
in the same framework.  The biggest challenge here is likely to be
incorporating the notion of time, and so it may be best to first try
and represent a simpler calculus such as TPL.

\subsection{DynamiTE}
\label{future:dynamite}

As we hinted at in chapter \ref{dynamite}, there are still a number of
areas we wish to explore in the future.  These include:

\begin{itemize}
\item The development of further \texttt{Evolver} implementations
  which implement a variety of execution semantics.  This gives the
  possibility of experimenting with additional constraints and levels
  of priority, which may later be formally adopted in the calculus.
\item Expanding the locality framework so that we actually move code
  and data between processes or even hosts via a network.
\item The addition of data to the clock signals, allowing them not
  only to act as phasing signals but also as a mechanism for broadcast
  data.
\item The development of a parser and/or graphical design tool so that
  processes can be constructed from algebraic terms or diagrams,
  rather than Java code.
\end{itemize}

There are also plenty of possibilities for further work in providing
more complex implementations of the plugin services such as
interprocess communication via web services and supporting other
process calculi.  Hopefully, third parties may also wish to get
involved in this area.

\subsubsection{The Abstract Nomadic Time Machine}

In \ref{safeambmachine}, we looked at the abstract machine PAN which
is used to provide an implementation of the safe ambients calculus
\cite{sangiorgi:safeambientsmachine}.  Our existing translation schema
from Nomadic Time to DynamiTE, given in Table
\ref{tab:translationschema1}, goes directly from the calculus to a
programming language, Java.  With an abstract machine in the same vein
as PAN, but for Nomadic Time, we could instead express the translation
in more general terms.  The syntactic constructs of Nomadic Time would
be translated into operations using the abstract machine, while
DynamiTE would no longer implement the calculus directly, but instead
provide an implementation of the abstract machine.

By providing this additional layer of abstraction, we have an
implementation of Nomadic Time that exists at a more formal level.
This allows us to prove properties of this implementation, without
having to contend with the complicated semantics of a programming
language such as Java.  It may also be possible to express other
calculi using the same abstract machine, thus giving a general
framework for implementations of process calculi.

\subsubsection{Implement Bigger Applications with DynamiTE}

Through the course of this thesis, we've shown how DynamiTE and its
associated development process can be used to take an application
through the usual stages of requirements analysis (\ref{app:req}),
design (\ref{app:nt}) and implementation (\ref{app:dynamite}).  For
this, we used the example of a music player with a fairly limited set
of requirements.  While this is fine as an easily digestable example
here, to really prove the utility of DynamiTE, we need to develop some
big examples using the framework and prove that it really helps with
concurrent applications on a scale where tracking down race conditions
and deadlocks starts to become intractable.

\subsection{The Type System}
\label{future:types}

\subsubsection{More Advanced Types}

There are a number of ideas that could be further investigated:

\begin{itemize}
\item The addition of types to actions and/or clocks.  In \ref{tnted},
  we explore one possibility for the former, but there may be more.
\item Bouncers only have a single rudimentary type; this could be
  extended.  Alternatively, the syntactic form of bouncers could be
  removed altogether, with the same functionality instead being
  provided by the type system.
\item It may be worth implementing the types in a theorem prover and
  deducing results, such as proofs of consistency.
\end{itemize}

\subsubsection{Nomadic Time Extended With Data}
\label{tnted}

At present, the calculus does not explicitly represent the passing of
data in any form.  When synchronisation occurs on a channel, $a$, it
simply produces a $\tau$ transition.  Within DynamiTE, the transfer of
a single \texttt{Object} from sender to recipient takes place.  This
may be inefficient if many objects need to be sent, and it would be
preferable to transfer multiple items in a single synchronisation.  It
would also be preferable if the type of the data was represented, so
that the type expected by the receiver matches that sent by the
sender.

Both of these could be implemented via the type system.  The matching
of the number of items being sent with those received could also be
performed through additions to the syntax and semantics, but also has
the downside of introducing a new set of names; we already have
distinct sets of names ($\names$), co-names ($\conames$) and clocks
($\timers$) along with the names used by environs.

Such an extension would draw on Milner's work on sorts (see
\ref{sorts}) and allow synchronisation to proceed only when the
following holds for $a : \vec{x}$ and $\overline{a} : \vec{y}$:

\begin{enumerate}
\item The length $n$ of $\vec{x}$ ($\#x$) must equal the length of
  $\vec{y}$ ($\#y$).
\item For each element $x_i$ in $\vec{x}$, the corresponding element
  in $\vec{y}$, $y_i$, must have the same type $t_i$.
\end{enumerate}

The types used in $\vec{x}$ and $\vec{y}$ are not members of $T$ (see
\ref{typesys}), but are user-defined.  In DynamiTE, for example, $t_1$
may be \texttt{String}.  Thus if $a$ is defined as having the type
$(t_1, t_2, t_3)$, it can only synchronise with $\overline{a}$ if it
also has the type $(t_1, t_2, t_3)$.  Clearly, the second rule could
be widened to allow a subtyping relation between the two types, rather
than equivalence.

\subsubsection{Explicit Type Assignment}

At present, there is no explicit assignment of types to processes in
the type system.  We assume that the user either adds their own axioms
to fix the type of a process, or that the type of a process is
derived; for example, we can derive a suitable type for
$\tntin{m}.\nil$ through the knowledge that $m \in g(\mathscr{R})$
must hold.  In the DynamiTE implementation of the type system (see
\ref{dyn:type}), types are assigned when constructing \texttt{Nil},
\texttt{Delta}, \texttt{Stall} and \texttt{Var} instances.  This could
also be added to the formal version by modifying the syntax and
semantics to use $\nil : T$, $\Delta : T$ and $\Delta_\sigma : T$.

\subsection{Other Applications}
\label{future:apps}

There may be other domains in which our calculus will find
applicability.  Pervasive computing may be one such area, as may web
services and biological modelling.  In addition, its relation to P
systems (\ref{psystems}) and the ability to encode it using a bigraph
(\ref{bigraphs}) would form interesting areas of study.
