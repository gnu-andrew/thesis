\chapter{Future Work}
\label{futurework}

This chapter briefly discusses future work on the calculus, with regard
to both the theoretical aspects (sections \ref{futuretypes} and
\ref{futureequiv}) and its applications (section \ref{futureapp}).

\section{The Type System}
\label{futuretypes}

A simple type system presented in \ref{typesys} has been completed.
One issue remains; how are groups to be allocated to processes?  It may be
necessary to add a form of type annotation or a group binder (e.g. ($\nu
G$); see \ref{ambienttypes}), as in the type systems based on the
ambient calculus.

On a more general note, it would be interesting to consider further the
typing of both actions and clocks.  At present, they just have the
simplest of types, but this could be extended to provide more
interesting behaviour.  Similarly, bouncers only have rudimentary types
and may possibly be changed to include more detail.

\section{Equivalence}
\label{futureequiv}

The main element lacking in the current version of TNT is some notion of
equivalence.  An equivalence notion is necessary to be able to compare
processes, with the additional benefit of being able to reduce them to a
simpler form.

Any bisimulation theory for TNT will be based on the labelled transition
system defined by the semantics.  In particular, the semantics share a
lot in common with those of CaSE, for which a form of bisimulation-based
equivalence (temporal observation congruence) already exists.  With
respect to the changes involved in TNT, the additional transitions
provided by the mobility primitives are all silent actions, which are
discarded by an observation-based congruence.  As a result, it is
unclear what changes will actually be necessary to provide the same
behavioural theory for TNT as is defined by this congruence for CaSE.

Mobility affects the topology of the process, and thus the difference
between two processes which exhibit mobility is much clearer via a
structural comparison.  Hence, structural congruence may end up being
more important with respect to the addition of mobility.

\section{Applications}
\label{futureapp}

The primary application of TNT will be biological modelling.  This is an
area where physical location is important and mobility is realised as
shifts in topological composition.  Nested structures are important,
being clearly present in, for example, cell membranes.  Within the
literature, models of a similar style have already been applied,
including Cardelli's brane calculi \cite{brane04} and the membrane
computing of P systems (see section \ref{psystems}).  This will also be
an interesting area to consider in terms of diversity, providing a
pleasant contrast to the more theoretical notions implicit in
designing the calculus itself.

One particular case study that is already being considered is that of
quorum sensing (see section \ref{bioapps}).  In brief, this
focuses on the reactive behaviour of bacteria in relation to the current
level of concentration of a particular gene.  The multi-party
synchronisation implicit in TNT is likely to be useful in this area.

In modelling this, it is expected that additions will be made to the
calculus to allow the use of probabilistic derivation, akin to the use
of the stochastic $\pi$ calculus and the Gillespie algorithm in similar
studies (see section \ref{bioapps}).  In particular, this will hopefully
allow simulations to be run and the behaviour of the model to be
analysed.

In the longer term, probably outside the scope of this thesis, there may
be other domains in which TNT will find applicability.  Pervasive
computing is thought to be one such area, and this may make more
effective use of the type system than biological modelling.  In
addition, its relation to P systems (\ref{psystems}) and the ability to
encode it using a bigraph (\ref{bigraphs}) would form interesting areas
of study.