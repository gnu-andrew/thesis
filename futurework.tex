\chapter{Future Work}
\label{futurework}

\section{Introduction}

This chapter briefly discusses future work on the calculus, with regard
to both the theoretical aspects (sections \ref{future:types} and
\ref{future:equiv}) and its applications (section \ref{future:dynamite}).

\section{The Type System}
\label{future:types}

It would be interesting to consider further the
typing of both actions and clocks.  At present, they just have the
simplest of types, but this could be extended to provide more
interesting behaviour.  Similarly, bouncers only have rudimentary types
and may possibly be changed to include more detail.

\subsection{Nomadic Time Extended with Data}
\label{tnted}
Chapter \ref{tnted} describes an optional data extension to the
calculus which allows channels to also be typed.
The addition of explicit data provision to the calculus via the
  use of typed channels -- see \ref{tnted}.

\section{Equivalence}
\label{future:equiv}

The main element lacking in the current version of TNT is some notion of
equivalence.  An equivalence notion is necessary to be able to compare
processes, with the additional benefit of being able to reduce them to a
simpler form.

Any bisimulation theory for TNT will be based on the labelled transition
system defined by the semantics.  In particular, the semantics share a
lot in common with those of CaSE, for which a form of bisimulation-based
equivalence (temporal observation congruence) already exists.  With
respect to the changes involved in TNT, the additional transitions
provided by the mobility primitives are all silent actions, which are
discarded by an observation-based congruence.  As a result, it is
unclear what changes will actually be necessary to provide the same
behavioural theory for TNT as is defined by this congruence for CaSE.

Mobility affects the topology of the process, and thus the difference
between two processes which exhibit mobility is much clearer via a
structural comparison.  Hence, structural congruence may end up being
more important with respect to the addition of mobility.

\section{DynamiTE}
\label{future:dynamite}

To conclude, we have presented the overall structure of the DynamiTE
framework for concurrent systems, with particular note to its more
interesting aspects involving the use of signalling (via clock ticks)
and process migration.  We have also outlined its underlying theoretical
basis in the form of the process calculus TNT, further details of which
are provided in the cited references.

We believe that the framework provides a unique way of developing
concurrent systems.  It provides features which have already proved
advantageous in a theoretical setting, such as the n-ary process
synchronisation mechanism described in chapter \ref{globsync}.  The
existence of a formal theory for DynamiTE's behaviour gives many
advantages over more ad-hoc approaches, allowing the underlying
mechanisms to be rigorously examined before being applied to the
implementation.  For example, the equivalence of two processes may be
established clearly and unambiguously in the underlying process
calculus and then used to compare the implementation of a process to
its specification.

The DynamiTE framework is still in heavy development.  At its lowest
level, it provides a means of simulating the operations of the TNT
process calculus, allowing them to be more clearly understood.  In
application, it can provide a useful mechanism for structuring
concurrent programs, clearly dividing internal behaviour and
interprocess communication.  The presence of signalling and code
migration also means that fairly complex concepts can be leveraged by
the programmer in the simple manner provided by the framework.

There are still areas we wish to explore in the future.  One such
proposition is the addition of data to the clock signals, allowing
them not only to act as phasing signals but also as a mechanism for
broadcast data.  It would also be interesting to further expand on the
plugin frameworks mentioned, by providing more complex implementations
such as interprocess communication via web services.

\section{Other Applications}
there may
be other domains in which TNT will find applicability.  Pervasive
computing is thought to be one such area, and this may make more
effective use of the type system than biological modelling.  In
addition, its relation to P systems (\ref{psystems}) and the ability to
encode it using a bigraph (\ref{bigraphs}) would form interesting areas
of study.

\section{Conclusion}
