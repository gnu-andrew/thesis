\chapter{Progress}
\label{progress}

A well-typed term is not stuck.  It can either take a step according
to the operational semantics or is one of $\Delta$, $\Omega$ or $X$.

\begin{theorem}
If $\Gamma \vdash P : t : T$ then either $P \derives{} P'$, $P =
\Delta$, $P = \Omega$ or $P = X$.
\end{theorem}

\begin{proof}
By induction on a derivation of $P:t$.  At each step, we assume that
the desired property holds for all subderivations and proceed by case
analysis on the final rule in the derivation.  We assume that
$\timers$ is non-empty; if instead $\timers = \emptyset$ then
$\Delta_\sigma$ and $\nil$ are equivalent to $\Delta$ and are possible
values for $P$.

\case{T-Nil}:

\noindent If the last rule in the derivation is T-Nil, then we know
from the form of the rule that $P$ must be the process term $\nil$.
From the $Idle$ rule in the semantics, $\nil$ has a transition for
each clock $\sigma \in \timers$ and so is not stuck.

\case{T-Stop}:

\noindent If the last rule in the derivation is T-Stop, then we know
from the form of the rule that $P$ must be the term $\Delta$.

\case{T-Stall}:

\noindent If the last rule in the derivation is T-Stall, then we know
from the form of the rule that $P$ must be the process term
$\Delta_\sigma$.  From the $Stall$ rule in the operational semantics,
$\Delta_\sigma$ has a transition for each clock $\rho$ where $\rho \ne
\sigma$ and so is not stuck.

\case{T-Var}:

\noindent If the last rule in the derivation is T-Var, then we know
from the form of the rule that $P$ must be the process term $X$.

\case{T-Act}:

\noindent If the last rule in the derivation is T-Act, then we know
from the form of the rule that $P$ must be the process term
$\alpha.E$.  The operational semantics provide two rules, $Act$ and
$Patient$, so that $\alpha.E$ has a transition $\alpha.E
\derives{\alpha} E$ and additional transitions $\alpha.E
\derives{\sigma} \alpha.E$ for each clock $\sigma \in \timers$.  So
the term is not stuck.

\case{T-Rec}:

\noindent If the last rule in the derivation is T-Rec, then we know
from the form of the rule that $P$ must be the process term $\mu X.E$.
From the rule $Rec$ in the operational semantics, $\mu X.E$ has a
transition $\derives{\gamma}$ for each such transition that occurs
from $E$.  By the induction hypothesis, we know $E$ is not stuck so
$\mu X.E$ is also not stuck as it either has the same transitions, or
$E = \Delta$, $E = \Omega$ or $E = X$.

\case{T-Res}:

\noindent If the last rule in the derivation is T-Res, then we know
from the form of the rule that $P$ must be the process term $E
\res{a}$.  From the rule $Res$ in the semantics, $E \res{a}$ has a
transition $\derives{\gamma}$ for each such transition from $E$, minus
those where $\gamma = a$.  We know by the induction hypothesis that
$E$ is not stuck.  If $E$ has transitions, then $E \res{a}$ also has
transitions as $a$ will never match cases where $\gamma \in \timers$
(produced by the base rules $Idle$ and $Patient$) or $\gamma \in
\highpris$ (produced by $Par3$, $LHid1$, $InEnv$, $OutEnv$, $Open$,
$ProcIn$ and $ProcOut$).  Otherwise, $E = \Delta$, $E = \Omega$ or $E
= X$.  In either case, $E \res{a}$ is not stuck.

\case{T-Sum}:

\noindent If the last rule in the derivation is T-Sum, then we know
from the form of the rule that $P$ must be the process term $E + F$.
By the induction hypothesis, neither $E$ or $F$ are stuck and the
rules $Sum1$ and $Sum2$ cause any transitions from the composed
processes to also hold for $E + F$.  Thus either $E + F$ has
transitions, or both $E$ and $F$ are one of $\Omega$, $X$ or $\Delta$.
For both, $E + F$ is not stuck.

\case{T-Par}:

\noindent If the last rule in the derivation is T-Par, then we know
from the form of the rule that $P$ must be the process term $E \pc F$.
The same logic applies as for the summation case using the rules
$Par1$ and $Par2$.  Using the rules $Par3$, $InEnv$, $OutEnv$, $Open$,
$ProcIn$ and $ProcOut$, $E \pc F$ can also produce further
transitions.  So $E \pc F$ is not stuck.

\case{T-FTO}:

\noindent If the last rule in the derivation is T-FTO, then we know
from the form of the rule that $P$ must be the process term
$\timeout{E}{\sigma}{F}$.  Again, we know $E$ and $F$ are not stuck
from the induction hypothesis.  All transitions $\derives{\gamma}$
produced by $E$ hold for $\timeout{E}{\sigma}{F}$ with the exception
of the case where $\gamma = \sigma$.  In this case, there is a
transition $\timeout{E}{\sigma}{F} \derives{\sigma} F$.  If $E = X$ or
$E = \Delta$, then the transition $\timeout{E}{\sigma}{F}
\derives{\sigma} F$ is still present, as $E$ has no transitions and
thus $E \nderives{h}$.  Therefore, $\timeout{E}{\sigma}{F}$ is not
stuck as it always has transitions.

\case{T-STO}:

\noindent If the last rule in the derivation is T-STO, then we know
from the form of the rule that $P$ must be the process term
$\stimeout{E}{\sigma}{F}$.  The proof here is the same as for T-FTO,
the only difference between the two being that $FTO2$ is split into
the two rules $STO2$ and $STO3$, so that the timeout is preserved for
$\derives{\rho}$ where $\rho \ne \sigma$.

\case{BNil}:

\noindent If the last rule in the derivation is BNil, then we know
from the form of the rule that $P$ must be the process term $\Omega$.

\case{BRec}:

\noindent If the last rule in the derivation is BRec, then we know
from the form of the rule that $P$ must be the process term $\mu X.B$
and the same proof holds as for T-Rec using the $Rec$ rule from the
operational semantics.

\case{BIn}:

\noindent If the last rule in the derivation is BIn, then we know from
the form of the rule that $P$ must be the process term $\bin.B$.  The
operational semantics provide two rules, $Cap1$ and $Cap2$, so that
$\ambop.E$ has a transition $\ambop.E \derives{\ambop} E$ and
additional transitions $\ambop.E \derives{\sigma} \ambop.E$ for each
clock $\sigma \in \timers$.  So the term is not stuck.

\case{BOut}:

\noindent If the last rule in the derivation is BOut, then we know
from the form of the rule that $P$ must be the process term $\bout.B$.
The same proof applies as for BIn.

\case{BOpen}:

\noindent If the last rule in the derivation is BOpen, then we know
from the form of the rule that $P$ must be the process term
$\bopen.B$.  The same proof applies as for BOpen.

\case{BSum}:

\noindent If the last rule in the derivation is BSum, then we know
from the form of the rule that $P$ must be the process term $B + B'$.
The same proof applies as for T-Sum, using the operational semantics
rule $Sum$.

\case{T-Environ}:

\noindent If the last rule in the derivation is Environ, then we know
from the form of the rule that $P$ must be the process term
$\locv{m}{E}{B}{\vec{\sigma}}$.  The operational semantics provide
three rules, $LHd1$, $LHd2$ and $LHd3$, so that
$\locv{m}{E}{B}{\vec{\sigma}}$ has a transition
$\locv{m}{E}{B}{\vec{\sigma}} \derives{h} E$ for each transition
$\derives{h}$ produced by $E$ and additional transitions $\ambop.E
\derives{\rho} \ambop.E$ for each clock $\rho \in \timers$ where $\rho
\not \in \vec{\sigma}$.  Any transitions $E \derives{\sigma} E'$,
where $\sigma \in \vec{\sigma}$ convert to transitions of the form
$\locv{m}{E}{B}{\vec{\sigma}} \derives{\tau}
\locv{m}{E'}{B}{\vec{\sigma}}$.  Transitions produced by the rules
$Act$ and $Cap1$ are not propogated, but the assumed presence of
clocks means that at least one transition will always apply to
$\locv{m}{E}{B}{\vec{\sigma}}$, so the term is not stuck.

\case{T-EnvIn}:

\noindent If the last rule in the derivation is T-EnvIn, then we know
from the form of the rule that $P$ must be the process term
$\locv{n}{\tntin{m}.E}{B}{\vec{\sigma}}$.  As with the cases BIn, BOut
and BOpen, transitions arise from the rules $Cap1$ and $Cap2$, so the
term is not stuck.

\case{T-EnvOut}:

\noindent If the last rule in the derivation is T-EnvOut, then we know
from the form of the rule that $P$ must be the process term
$\locv{n}{\locv{m}{\locv{k}{\tntin{m}.E}{B_1}{\vec{\sigma}}}{B_2}{\vec{\rho}}}{B_3}{\vec{\gamma}}$.
The proof is the same as for T-EnvIn, BIn, BOut and BOpen.

\case{T-Open}:

\noindent If the last rule in the derivation is T-Open, then we know
from the form of the rule that $P$ must be the process term
$\locv{n}{\tntopen{m}.E \pc
  \locv{m}{F}{B_1}{\vec{\sigma}}}{B_2}{\vec{\rho}}$.  The same proof
applies as for T-EnvIn and T-EnvOut, with the addition of the
transition $P \derives{\topen} \locv{n}{E \pc F}{B_2}{\vec{\rho}}$
resulting from the $Open$ rule which causes the environ $m$ to be
destroyed.

\case{T-ProcIn}:

\noindent If the last rule in the derivation is ProcIn, then we know
from the form of the rule that $P$ must be the process term $a.E \pc
\procin{a}{m}.F$.  In a similar manner to T-Open, transitions are
produced by the $Cap1$ and $Cap2$ rules in addition to the possibility
of process movement arising from the $ProcIn$ rule, $P \derives{\tin}
\locv{m}{E}{B}{\vec{\sigma}} \pc F$ so $P$ is not stuck.

\case{T-ProcOut}:

\noindent If the last rule in the derivation is T-ProcOut, then we
know from the form of the rule that $P$ must be the process term
$\locv{n}{\locv{m}{a.E \pc
    \procout{a}{m}.F}{B_1}{\vec{\sigma}}}{B_2}{\vec{\rho}}$.  As with
T-ProcIn, transitions are produced by $Cap1$ and $Cap2$ rules in
addition to the possibility of process movement arising from the
$ProcOut$ rules, $P \derives{\tout} \locv{n}{a.E \pc
  \locv{m}{F}{B_1}{\vec{\sigma}}}{B_2}{\vec{\rho}}$, so $P$ is not stuck.

\end{proof}
