\chapter{Type Preservation}

If a well-typed term takes a step of evaluation, then the resulting
term is also well-typed.

\begin{proposition}
If $\Gamma \vdash P : t : T$ and $P \derives{} P'$, then $\Gamma \vdash (\exists t' : T) P' : t'$
\end{proposition}

\begin{proof}
By induction on a derivation of $P:t$.  At each step, we assume that
the desired property holds for all subderivations and proceed by case
analysis on the final rule in the derivation.

\case{T-Nil}: $P = \nil$, $t = g$

\noindent If the last rule in the derivation is $T-Nil$, then we know
from the form of the rule that $P$ must be the process term $\nil$ and
$t$ must be the type $g$ where $g : Group$.  From the semantics
($Idle$), $\nil$ has a transition $\nil \derives{\sigma} \nil$ for
each clock $\sigma$ in $\timers$.  In each case, $P' = \nil$ which, by
the $T-Nil$ rule, has type $g$ so our proposition is satisfied.

\case{T-Stop}: $P = \Delta$, $t = g$

\noindent If the last rule in the derivation is $T-Stop$, then we know
from the form of the rule that $P$ must be the process term $\Delta$
and $t$ must be the type $g$ where $g : Group$.  From the semantics,
there are no transitions for $\Delta$ so there is nothing to prove.

\case{T-Stall}: $P = \Delta_\sigma$, $t = g$

\noindent If the last rule in the derivation is $T-Stall$, then we know
from the form of the rule that $P$ must be the process term
$\Delta_\sigma$ and $t$ must be the type $g$ where $g : Group$.  From
the semantics ($Stall$), $\Delta_\sigma$ has a transition
$\Delta_\sigma \derives{\rho} \Delta_\sigma$ for each clock $\rho$ in
$\timers$ where $\rho \ne \sigma$.  In each case, $P' = \Delta_\sigma$
which, by the $T-Stall$ rule, has type $g$ so our proposition is
satisfied.

\case{T-Var}: $P = X$, $t = g$

\noindent If the last rule in the derivation is $T-Var$, then we know
from the form of the rule that $P$ must be the process term $X$ and
$t$ must be some type $t$.  From the semantics, there are no
transitions for $X$ so there is nothing to prove.

\case{T-Act}: $P = \alpha.E$, $t = g$

\noindent If the last rule in the derivation is $T-Act$, then we know
from the form of the rule that $P$ must be the process term
$\alpha.E$ and $t$ must be the type $g$ where $g : Group$.  From
the semantics, there are two subcases:

\subcase{Act}: $P' = E$, $t' = g$

\noindent The $Act$ rule from the semantics provides the transition
$\alpha.E \derives{\alpha} E$, so $P'$ is $E$.  From the
subderivations of the $T-Act$ typing rule, we know that $E : g :
Group$ so we can apply the induction hypothesis to obtain $P' : g$.

\subcase{Patient}: $P' = \alpha.E$, $t' = g$

\noindent From the $Patient$ rule of the semantics, $\alpha.E$ has a transition
$\alpha.E \derives{\sigma} \alpha.E$ for each clock $\sigma$ in
$\timers$.  In each case, $P' = \alpha.E$ which, by the $T-Act$ typing
rule, has type $g$ so our proposition is satisfied.

\case{T-Rec}: $P = \mu X.E$, $t = g$

\noindent If the last rule in the derivation is $T-Rec$, then we know
from the form of the rule that $P$ must be the process term $\mu X.E$
and $t$ must be the type $g$ where $g : Group$.  From the $Rec$ rule
of the semantics, $\mu X.E$ has a transition to $E'\{\mu X.E/X\}$ for
any transition $\gamma$ which can be performed by $E$.  In each case,
$P' = E'\{\mu X.E/X\}$ so we need to show that this is well-typed.
From the subderivations of the $T-Rec$ typing rule, we know that $E :
g : Group$ so we can apply the induction hypothesis to obtain $E' : g
: Group$.  So, we just need to show that the well-typedness of $E'$ is
preserved when the substitution ($\{\mu X.E/X\}$ is performed.

\begin{lemma}
If $\Gamma, x : S \vdash P : t$ and $\Gamma \vdash x : S$ then $\Gamma \vdash P\{x/X\} : t$
\end{lemma}

\begin{proof}
By induction on a derivation of the statement $\Gamma, x : S \vdash P
: t$.  For a given derivation, we proceed by cases on the final typing
rule.  There is only one case where $X$ appears and this is $Var$,
where $P = X$ and $t = g$.  There are two possible subcases:

\subcase{Match}: $x = X$

\noindent If $X$ matches the bound variable being substituted, $x$, then it
becomes $x$, which we know is well-typed from the precondition.
  
\subcase{NoMatch}: $x \ne X$

\noindent If $X$ does not match the bound variable being substituted, $x$, then it
remains as $X$, which is typeable using the $T-Var$ rule.

\end{proof}

\case{T-Res}: $P = E \res{a}$, $t = g$

\noindent If the last rule in the derivation is $T-Res$, then we know
from the form of the rule that $P$ must be the process term $E
\res{a}$ and $t$ must be the type $g$ where $g : Group$.  From the
$Res$ rule of the semantics, $E \res{a}$ has a transition to $E'
\res{a}$ for all transitions ($\gamma$) which can be performed by $E$
where $\gamma \ne a$.  In each case, $P' = E'$ so we need to show that
this is well-typed.  From the subderivations of the $T-Res$ typing
rule, we know that $E : g : Group$ so we can apply the induction
hypothesis to obtain $E' : g : Group$.

\case{T-Sum}: $P = E + F$, $t = g \oplus g'$

\noindent If the last rule in the derivation is $T-Sum$, then we know
from the form of the rule that $P$ must be the process term $E + F$
and $t$ must be the type $g \oplus g'$ where $g, g' : Group$.  From
the semantics, there are two subcases:

\subcase{Sum1}: $P' = E'$, $t' = g$

\noindent The $Sum1$ rule from the semantics provides the transition
$E + F \derives{\kappa} E$, so $P'$ is $E'$.  From the
subderivations of the $T-Sum$ typing rule, we know that $E : g : Group$
so we can apply the induction hypothesis to obtain $P' : g$.

\subcase{Sum2}: $P' = E' + F'$, $t' = g$

\noindent From the $Sum2$ rule of the semantics, $E + F$ has a
transition $E + F \derives{\sigma} E' + F'$ for each clock $\sigma$ in
$\timers$.  In each case, $P' = E' + F'$.  From the subderivations of
the $T-Sum$ typing rule, we know that $E : g : Group$ and $F : g :
Group$ and by applying the induction hypothesis, we know both $E'$ and
$F'$ are well typed.  As a result, we can apply $T-Sum$ to give $P' : g
\oplus g'$.

\case{T-Par}: $P = E | F$, $t = g \otimes g'$

\noindent If the last rule in the derivation is $T-Par$, then we know
from the form of the rule that $P$ must be the process term $E \pc F$
and $t$ must be the type $g \otimes g'$ where $g, g' : Group$.  From
the semantics, there are eight subcases:

\subcase{Par1}: $P' = E' \pc F$, $t' = g$

\noindent The $Par1$ rule from the semantics provides the transition
$E \pc F \derives{\kappa} E' \pc F$, so $P'$ is $E' \pc F$.  From the
subderivations of the $T-Par$ typing rule, we know that $E : g : Group$
and $F : g : Group$ so we can apply the induction hypothesis to obtain
$E' : g$ and the $T-Par$ typing rule to give $E' \pc F : g \otimes g'$.

\subcase{Par2}: $P' = E' \pc F'$, $t' = g$

\noindent From the $Par2$ rule of the semantics, $E \pc F$ has a
transition $E \pc F \derives{\tau} E' \pc F'$ when $E$ and $F$
synchronise.  In this case, $P' = E' + F'$.  From the subderivations
of the $T-Par$ typing rule, we know that $E : g : Group$ and $F : g :
Group$ and by applying the induction hypothesis, we know both $E'$ and
$F'$ are well typed.  As a result, we can apply $T-Par$ to give $P' : g
\oplus g'$.

\subcase{Par3}: $P' = E' \pc F'$, $t' = g$

\noindent From the $Par3$ rule of the semantics, $E \pc F$ has a
transition $E \pc F \derives{\sigma} E' \pc F'$ for each clock
$\sigma$ in $\timers$, as long as $E \pc F$ does not contain a high
priority transition ($\nderives{h}$).  In each case, $P' = E' + F'$.
From the subderivations of the $T-Par$ typing rule, we know that $E : g
: Group$ and $F : g : Group$ and by applying the induction hypothesis,
we know both $E'$ and $F'$ are well typed.  As a result, we can apply
$T-Par$ to give $P' : g \oplus g'$.

\case{BNil}: $P = \Omega$, $t = Bouncer$

\noindent If the last rule in the derivation is $BNil$, then we know
from the form of the rule that $P$ must be the process term $\Omega$
and $t$ must be the type $Bouncer$.  From the semantics, there are no
transitions for $\Omega$ so there is nothing to prove.

\end{proof}

