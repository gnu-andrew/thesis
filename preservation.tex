\chapter{Preservation}
\label{preservation}

If a well-typed term takes a step of evaluation, then the resulting
term is also well-typed.

\begin{theorem}
If $\Gamma \vdash P : t : T$ and $P \derives{} P'$, then there exists
$t' : T$ such that $\Gamma \vdash P' : t'$.
\end{theorem}

\begin{proof}
By induction on a derivation of $P:t$.  At each step, we assume that
the desired property holds for all subderivations and proceed by case
analysis on the final rule in the derivation.

\case{T-Nil}:

\noindent If the last rule in the derivation is T-Nil, then we know
from the form of the rule that $P$ must be the process term $\nil$ and
$t$ must be the type $g$ where $g : Group$.  From the semantics
($Idle$), $\nil$ has a transition $\nil \derives{\sigma} \nil$ for
each clock $\sigma$ in $\timers$.  In each case, $P' = \nil$ which, by
the T-Nil rule, has type $g$ so our proposition is satisfied.

\case{T-Stop}:

\noindent If the last rule in the derivation is T-Stop, then we know
from the form of the rule that $P$ must be the process term $\Delta$
and $t$ must be the type $g$ where $g : Group$.  From the semantics,
there are no transitions for $\Delta$ so there is nothing to prove.

\case{T-Stall}:

\noindent If the last rule in the derivation is T-Stall, then we know
from the form of the rule that $P$ must be the process term
$\Delta_\sigma$ and $t$ must be the type $g$ where $g : Group$.  From
the semantics ($Stall$), $\Delta_\sigma$ has a transition
$\Delta_\sigma \derives{\rho} \Delta_\sigma$ for each clock $\rho$ in
$\timers$ where $\rho \ne \sigma$.  In each case, $P' = \Delta_\sigma$
which, by the T-Stall rule, has type $g$ so our proposition is
satisfied.

\case{T-Var}:

\noindent If the last rule in the derivation is T-Var, then we know
from the form of the rule that $P$ must be the process term $X$ and
$t$ must be some type $t$.  From the semantics, there are no
transitions for $X$ so there is nothing to prove.

\case{T-Act}:

\noindent If the last rule in the derivation is T-Act, then we know
from the form of the rule that $P$ must be the process term
$\alpha.E$ and $t$ must be the type $g$ where $g : Group$.  From
the semantics, there are two subcases:

\subcase{Act}:

\noindent The $Act$ rule from the semantics provides the transition
$\alpha.E \derives{\alpha} E$, so $P'$ is $E$.  From the
subderivations of the T-Act typing rule, we know that $E : g :
Group$ so we can apply the induction hypothesis to obtain $P' : g$.

\subcase{Patient}:

\noindent From the $Patient$ rule of the semantics, $\alpha.E$ has a transition
$\alpha.E \derives{\sigma} \alpha.E$ for each clock $\sigma$ in
$\timers$.  In each case, $P' = \alpha.E$ which, by the T-Act typing
rule, has type $g$ so our proposition is satisfied.

\case{T-Rec}:

\noindent If the last rule in the derivation is T-Rec, then we know
from the form of the rule that $P$ must be the process term $\mu X.E$
and $t$ must be the type $g$ where $g : Group$.  From the $Rec$ rule
of the semantics, $\mu X.E$ has a transition to $E'\{\mu X.E/X\}$ for
any transition $\gamma$ which can be performed by $E$.  In each case,
$P' = E'\{\mu X.E/X\}$ so we need to show that this is well-typed.
From the subderivations of the T-Rec typing rule, we know that $E :
g : Group$ so we can apply the induction hypothesis to obtain $E' : g
: Group$.  So, we just need to show that the well-typedness of $E'$ is
preserved when the substitution ($\{\mu X.E/X\}$ is performed.

\begin{lemma}
If $\Gamma, x : S \vdash P : t$ and $\Gamma \vdash x : S$ then $\Gamma \vdash P\{x/X\} : t$
\end{lemma}

\begin{proof}
By induction on a derivation of the statement $\Gamma, x : S \vdash P
: t$.  For a given derivation, we proceed by cases on the final typing
rule.  There is only one case where $X$ appears and this is $Var$,
where $P = X$ and $t = g$.  There are two possible subcases:

\subcase{Match}: $x = X$

\noindent If $X$ matches the bound variable being substituted, $x$, then it
becomes $x$, which we know is well-typed from the precondition.
  
\subcase{NoMatch}: $x \ne X$

\noindent If $X$ does not match the bound variable being substituted, $x$, then it
remains as $X$, which is typeable using the T-Var rule.

\end{proof}

\case{T-Res}:

\noindent If the last rule in the derivation is T-Res, then we know
from the form of the rule that $P$ must be the process term $E
\res{a}$ and $t$ must be the type $g$ where $g : Group$.  From the
$Res$ rule of the semantics, $E \res{a}$ has a transition to $E'
\res{a}$ for all transitions ($\gamma$) which can be performed by $E$
where $\gamma \ne a$.  In each case, $P' = E'$ so we need to show that
this is well-typed.  From the subderivations of the T-Res typing
rule, we know that $E : g : Group$ so we can apply the induction
hypothesis to obtain $E' : g : Group$.

\case{T-Sum}:

\noindent If the last rule in the derivation is T-Sum, then we know
from the form of the rule that $P$ must be the process term $E + F$
and $t$ must be the type $g \oplus g'$ where $g, g' : Group$.  From
the semantics, there are two subcases:

\subcase{Sum1}:

\noindent The $Sum1$ rule from the semantics provides the transition
$E + F \derives{\kappa} E'$, so $P'$ is $E'$.  From the
subderivations of the T-Sum typing rule, we know that $E : g : Group$
so we can apply the induction hypothesis to obtain $P' : g$.

\subcase{Sum2}:

\noindent From the $Sum2$ rule of the semantics, $E + F$ has a
transition $E + F \derives{\sigma} E' + F'$ for each clock $\sigma$ in
$\timers$.  In each case, $P' = E' + F'$.  From the subderivations of
the T-Sum typing rule, we know that $E : g : Group$ and $F : g :
Group$ and by applying the induction hypothesis, we know both $E'$ and
$F'$ are well typed.  As a result, we can apply T-Sum to give $P' : g
\oplus g'$.

\case{T-Par}:

\noindent If the last rule in the derivation is T-Par, then we know
from the form of the rule that $P$ must be the process term $E \pc F$
and $t$ must be the type $g \otimes g'$ where $g, g' : Group$.  From
the semantics, there are eight subcases:

\subcase{Par1}:

\noindent The $Par1$ rule from the semantics provides the transition
$E \pc F \derives{\kappa} E' \pc F$, so $P'$ is $E' \pc F$.  From the
subderivations of the T-Par typing rule, we know that $E : g : Group$
and $F : g' : Group$ so we can apply the induction hypothesis to obtain
$E' : g$ and the T-Par typing rule to give $E' \pc F : g \otimes g'$.

\subcase{Par2}:

\noindent From the $Par2$ rule of the semantics, $E \pc F$ has a
transition $E \pc F \derives{\tau} E' \pc F'$ when $E$ and $F$
synchronise.  In this case, $P' = E' + F'$.  From the subderivations
of the T-Par typing rule, we know that $E : g : Group$ and $F : g' :
Group$ and by applying the induction hypothesis, we know both $E'$ and
$F'$ are well typed.  As a result, we can apply T-Par to give $P' : g
\oplus g'$.

\subcase{Par3}:

\noindent From the $Par3$ rule of the semantics, $E \pc F$ has a
transition $E \pc F \derives{\sigma} E' \pc F'$ for each clock
$\sigma$ in $\timers$, as long as $E \pc F$ does not contain a high
priority transition ($\nderives{h}$).  In each case, $P' = E' + F'$.
From the subderivations of the T-Par typing rule, we know that $E : g
: Group$ and $F : g : Group$ and by applying the induction hypothesis,
we know both $E'$ and $F'$ are well typed.  As a result, we can apply
T-Par to give $P' : g \oplus g'$.

\subcase{InEnv}:

\noindent From the $InEnv$ rule of the semantics, $E \pc F$ has a
transition $E \pc F \derives{\tin} E' \pc F'$ where $E$ takes the form
$\locv{n}{G}{B_2}{\vec{\sigma}}$ and $F$ is
$\locv{m}{H}{B_1}{\vec{\rho}}$.  Following the transition, $P' =
\locv{m}{H \pc \locv{n}{G'}{B_2}{\vec{\sigma}}}{B'_1}{\vec{\rho}}$.
If this is well-typed, then, by using T-Environ, we know $H \pc
\locv{n}{G'}{B_2}{\vec{\sigma}}$ has type $g : Group$, $B'_1 :
Bouncer$ and $m \in \Resides{g}$.  According to T-Par, $g = g'
\otimes g''$, so $H$ has type $g'$ and $locv{n}{G'}{B_2}{\vec{sigma}}$
has type $g''$.  Again, T-Environ gives $G' : g''' : Group$, $B_2 :
Bouncer$ and $n \in \Resides{g'''}$.

From the subderivations of the T-Par typing rule, we know that $E :
g : Group$ and $F : g' : Group$ and by applying the induction
hypothesis, we know both $G'$ and $B'_1$ are well typed.  As $F$ is
well-typed, then $H$ must be well-typed by the rule T-Environ.  Thus
$P'$ is well-typed.

\subcase{OutEnv}:

\noindent From the $OutEnv$ rule of the semantics, $E \pc F$ has a
transition $E \pc F \derives{\tout} E' \pc F'$ where $E$ takes the
form $H$ and $F$ is $\locv{n}{G}{B_2}{\vec{\sigma}}$.  Following the
transition, $P' = \locv{n}{G'}{B_2}{\vec{\sigma}} \pc
\locv{m}{H}{B'_1}{\vec{\rho}}$.  Using T-Par, we know that $P' : g
  \otimes g'$, $\locv{n}{G'}{B_2}{\vec{\sigma}}$ is of type $g$ and
    $\locv{m}{H}{B'_1}{\vec{\rho}}$ is of type $g'$.

From the subderivations of the T-Par typing rule, we know that $E :
g : Group$ and $F : g' : Group$ and by applying the induction
hypothesis, we know both $G'$ and $B'_1$ are well typed.  By
T-Environ, $\locv{n}{G'}{B_2}{\vec{\sigma}}$ is well-typed as $G'$
and $B_2$ (from the original $F$) are well typed.  Similarly,
$\locv{m}{H}{B'_1}{\vec{\rho}}$ is well-typed as $H$ is well-typed
(from the original $E$) and $B'_1$ is well-typed.  Thus $P'$ is
well-typed.

\subcase{Open}:

\noindent From the $Open$ rule of the semantics, $E \pc F$ has a
transition $E \pc F \derives{\topen} E' \pc F'$ where $E$ takes the
form $G$ and $F$ is $\locv{m}{H}{B_1}{\vec{\sigma}}$.  Following the
transition, $P' = \locv{n}{G' \pc H}{B_2}{\vec{\sigma} \cup
  \vec{\rho}}$.  Using T-Par, we know that $P' : g \otimes g'$, $G'$
is of type $g$ and $H$ is of type $g'$.

From the subderivations of the T-Par typing rule, we know that $E : g
: Group$ and $F : g' : Group$ and by applying the induction
hypothesis, we know both $G'$ and $B'_1$ are well typed.  By
T-Environ, as $\locv{m}{H}{B_1}{\vec{\sigma}}$ is well-typed, $H$
(from the original $F$) is also well typed. Thus $P'$ is
well-typed.

\subcase{ProcIn}:

\noindent From the $ProcIn$ rule of the semantics, $E \pc F$ has a
transition $E \pc F \derives{\tin} E' \pc F'$.  Following the
transition, $P' = \locv{m}{E'}{B'_1}{\vec{\sigma}} \pc F'$.  Using
T-Par, we know that $P' : g \otimes g'$,
$\locv{m}{E'}{B'_1}{\vec{\sigma}}$ is of type $g$ and $F'$ is of type
$g'$.

From the subderivations of the T-Par typing rule, we know that $E : g
: Group$ and $F : g' : Group$ and by applying the induction
hypothesis, we know both $E'$, $F'$ and $B'_1$ are well typed.  By
T-Environ, $\locv{m}{E'}{B'_1}{\vec{\sigma}}$ is well-typed as both
$E'$ and $B'_1$ are.  Thus $P'$ is well-typed.

\subcase{ProcOut}:

\noindent From the $ProcOut$ rule of the semantics, $E \pc F$ has a
transition $E \pc F \derives{\tin} E' \pc F'$.  Following the
transition, $P' = E' \pc \locv{m}{F'}{B'_1}{\vec{\sigma}}$.  Using
T-Par, we know that $P' : g \otimes g'$,
$E'$ is of type $g$ and $\locv{m}{F'}{B'_1}{\vec{\sigma}}$ is of type
$g'$.

From the subderivations of the T-Par typing rule, we know that $E : g
: Group$ and $F : g' : Group$ and by applying the induction
hypothesis, we know both $E'$, $F'$ and $B'_1$ are well typed.  By
T-Environ, $\locv{m}{F'}{B'_1}{\vec{\sigma}}$ is well-typed as both
$F'$ and $B'_1$ are.  Thus $P'$ is well-typed.

\case{T-FTO}:

\noindent If the last rule in the derivation is T-FTO, then we know
from the form of the rule that $P$ must be the process term
$\timeout{E}{\sigma}{F}$ and $t$ must be the type $g \oplus g'$ where
$g, g' : Group$.  From the semantics, there are two subcases:

\subcase{FTO1}:

\noindent From the $FTO1$ rule of the semantics,
$\timeout{E}{\sigma}{F}$ has a transition $\timeout{E}{\sigma}{F}
\derives{\sigma} F$ for each clock $\sigma$ in $\timers$ as long as $E
\nderives{h}$.  In each case, $P' = F$.  From the subderivations of
the T-FTO typing rule, we know that $F : g : Group$ so $P'$ is thus
well-typed.

\subcase{FTO2}:

\noindent The $FTO2$ rule from the semantics provides the transition
$\timeout{E}{\sigma}{F} \derives{\gamma} E'$, so $P'$ is $E'$ where
$\gamma \ne \sigma$.  From the subderivations of the T-FTO typing
rule, we know that $E : g : Group$ so we can apply the induction
hypothesis to obtain $P' : g$.

\case{T-STO}:

\noindent If the last rule in the derivation is T-STO, then we know
from the form of the rule that $P$ must be the process term
$\stimeout{E}{\sigma}{F}$ and $t$ must be the type $g \oplus g'$ where
$g, g' : Group$.  From the semantics, there are three subcases:

\subcase{STO1}:

\noindent From the $STO1$ rule of the semantics,
$\stimeout{E}{\sigma}{F}$ has a transition $\stimeout{E}{\sigma}{F}
\derives{\sigma} F$ as long as $E \nderives{h}$.  So, $P' = F$.  From
the subderivations of the T-STO typing rule, we know that $F : g :
Group$ so $P'$ is thus well-typed.

\subcase{STO2}:

\noindent The $STO2$ rule from the semantics provides the transition
$\stimeout{E}{\sigma}{F} \derives{\kappa} E'$, so $P'$ is $E'$.  From
the subderivations of the T-STO typing rule, we know that $E : g :
Group$ so we can apply the induction hypothesis to obtain $P' : g$.

\subcase{STO3}:

\noindent From the $STO3$ rule of the semantics,
$\stimeout{E}{\sigma}{F}$ has a transition $\stimeout{E}{\sigma}{F}
\derives{\rho} F$ for each clock $\rho$ in $\timers$ as long as $E
\nderives{h}$ and $\rho \ne \sigma$.  In each case, $P' =
\stimeout{E'}{\sigma}{F}$.  From the subderivations of the T-STO
typing rule, we know that $E : g : Group$ and $F : g : Group$ so we
can apply the induction hypothesis to ensure $E'$ and $F'$ are well
typed. By then applying T-STO, we know that $P'$ is well-typed.

\case{BNil}:

\noindent If the last rule in the derivation is BNil, then we know
from the form of the rule that $P$ must be the process term $\Omega$
and $t$ must be the type $Bouncer$.  From the semantics, there are no
transitions for $\Omega$ so there is nothing to prove.

\case{BRec}:

\noindent If the last rule in the derivation is BRec, then we know
from the form of the rule that $P$ must be the process term $\mu X.B$
and $t$ must be the type $Bouncer$.  From the $Rec$ rule of the
semantics, $\mu X.B$ has a transition to $B'\{\mu X.B/X\}$ for any
transition $\gamma$ which can be performed by $B$.  In each case, $P'
= B'\{\mu X.B/X\}$ so we need to show that this is well-typed.  From
the subderivations of the T-Rec typing rule, we know that $B :
Bouncer$ so we can apply the induction hypothesis to obtain $B' :
Bouncer$.  We know from our previous lemma in the T-Rec case that the
well-typedness of $B'$ is preserved when the substitution ($\{\mu
X.B/X\}$ is performed, so $P'$ is well-typed.

\case{BIn}:

\noindent If the last rule in the derivation is BIn, then we know from
the form of the rule that $P$ must be the process term $\bin.B$ and
$t$ must be the type $Bouncer$.  From the semantics, there are two
subcases:

\subcase{Cap1}:

\noindent The $Cap1$ rule from the semantics provides the transition
$\ambop.E \derives{\ambop} E$, so $P'$ is $E$.  From the
subderivations of the BIn typing rule, we know that $B : Bouncer$ so
$P' : Bouncer$.

\subcase{Cap2}:

\noindent From the $Cap2$ rule of the semantics, $\ambop.E$ has a
transition $\ambop.E \derives{\sigma} \ambop.E$ for each clock
$\sigma$ in $\timers$.  In each case, $P' = \bin.E$ which is the same
as $P$ so our proposition is satisfied.

\case{BOut}:

\noindent If the last rule in the derivation is BOut, then we know from
the form of the rule that $P$ must be the process term $\bout.B$ and
$t$ must be the type $Bouncer$.  From the semantics, there are two
subcases:

\subcase{Cap1}:

\noindent The $Cap1$ rule from the semantics provides the transition
$\ambop.E \derives{\ambop} E$, so $P'$ is $E$.  From the
subderivations of the BOut typing rule, we know that $B : Bouncer$ so
$P' : Bouncer$.

\subcase{Cap2}:

\noindent From the $Cap2$ rule of the semantics, $\ambop.E$ has a
transition $\ambop.E \derives{\sigma} \ambop.E$ for each clock
$\sigma$ in $\timers$.  In each case, $P' = \bout.E$ which is the same
as $P$ so our proposition is satisfied.

\case{BOpen}:

\noindent If the last rule in the derivation is BOpen, then we know from
the form of the rule that $P$ must be the process term $\bopen.B$ and
$t$ must be the type $Bouncer$.  From the semantics, there are two
subcases:

\subcase{Cap1}:

\noindent The $Cap1$ rule from the semantics provides the transition
$\ambop.E \derives{\ambop} E$, so $P'$ is $E$.  From the
subderivations of the BOpen typing rule, we know that $B : Bouncer$ so
$P' : Bouncer$.

\subcase{Cap2}:

\noindent From the $Cap2$ rule of the semantics, $\ambop.E$ has a
transition $\ambop.E \derives{\sigma} \ambop.E$ for each clock
$\sigma$ in $\timers$.  In each case, $P' = \bopen.E$ which is the
same as $P$ so our proposition is satisfied.

\case{BSum}:

\noindent If the last rule in the derivation is BSum, then we know
from the form of the rule that $P$ must be the process term $B + B'$
and $t$ must be the type $Bouncer$.  From the semantics, there are two
subcases:

\subcase{Sum1}:

\noindent The $Sum1$ rule from the semantics provides the transition
$E + F \derives{\kappa} E'$, so $P'$ is $E'$.  From the
subderivations of the BSum typing rule, we know that $B : Bouncer$
so we can apply the induction hypothesis to obtain $P' : Bouncer$.

\subcase{Sum2}:

\noindent From the $Sum2$ rule of the semantics, $E + F$ has a
transition $E + F \derives{\sigma} E' + F'$ for each clock $\sigma$ in
$\timers$.  In each case, $P' = B'' + B'''$.  From the subderivations
of the BSum typing rule, we know that $B : Bouncer$ and $B' : Bouncer$
and by applying the induction hypothesis, we know both $B''$ and
$B'''$ are well typed.  As a result, we can apply BSum to give $P' :
Bouncer$.

\case{T-Environ}:

\noindent If the last rule in the derivation is Environ, then we know
from the form of the rule that $P$ must be the process term
$\locv{m}{E}{B}{\vec{\sigma}}$ and $t$ must be the type $g : Group$.
From the semantics, there are three subcases:

\subcase{LHd1}:

\noindent From the $LHd1$ rule of the semantics,
$\locv{m}{E}{B}{\vec{\sigma}}$ has a transition
$\locv{m}{E}{B}{\vec{\sigma}} \derives{\tau}
\locv{m}{E'}{B}{\vec{\sigma}}$ as long as $E \derives{\sigma} E'$ and
$\sigma \in \vec{\sigma}$.  So, $P' = \locv{m}{E'}{B}{\vec{\sigma}}$.
From the subderivations of the T-Environ typing rule, we know that $E
: g : Group$ and, by the induction hypothesis, $E'$ is also
well-typed.  By then applying T-Environ to the well-typed terms $E'$
and $B$, we can see that $P'$ is well-typed.

\subcase{LHd2}:

\noindent From the $LHd2$ rule of the semantics,
$\locv{m}{E}{B}{\vec{\sigma}}$ has a transition
$\locv{m}{E}{B}{\vec{\sigma}} \derives{h}
\locv{m}{E'}{B}{\vec{\sigma}}$.  So, $P' =
\locv{m}{E'}{B}{\vec{\sigma}}$.  From the subderivations of the
T-Environ typing rule, we know that $E : g : Group$ and, by the
induction hypothesis, $E'$ is also well-typed.  By then applying
T-Environ to the well-typed terms $E'$ and $B$, we can see that $P'$
is well-typed.

\subcase{LHd3}:

\noindent From the $LHd3$ rule of the semantics,
$\locv{m}{E}{B}{\vec{\sigma}}$ has a transition
$\locv{m}{E}{B}{\vec{\sigma}} \derives{\rho}
\locv{m}{E'}{B}{\vec{\sigma}}$ for each clock $\rho$ in $\timers$ as
long as $\rho \not \in \vec{\sigma}$ and $E \nderives{\sigma}$ where
$\sigma \in \vec{\sigma}$.  In each case, $P' =
\locv{m}{E'}{B}{\vec{\sigma}}$.  From the subderivations of the
T-Environ typing rule, we know that $E : g : Group$ and, by the
induction hypothesis, $E'$ is also well-typed.  By then applying
T-Environ to the well-typed terms $E'$ and $B$, we can see that $P'$
is well-typed.

\case{T-EnvIn}:

\noindent If the last rule in the derivation is T-EnvIn, then we know
from the form of the rule that $P$ must be the process term
$\locv{n}{\tntin{m}.E}{B}{\vec{\sigma}}$ and $t$ must be the type $g :
  Group$.  From the semantics, there are two subcases:

\subcase{Cap1}:

\noindent The $Cap1$ rule from the semantics provides the transition
$\ambop.E \derives{\ambop} E$, so $P'$ is
$\locv{n}{E}{B}{\vec{\sigma}}$.  From the subderivations of the
T-EnvIn typing rule, we know that $E : g : Group$ so by the T-Environ
rule $P'$ is well-typed.

\subcase{Cap2}:

\noindent From the $Cap2$ rule of the semantics, $\ambop.E$ has a
transition $\ambop.E \derives{\sigma} \ambop.E$ for each clock
$\sigma$ in $\timers$.  In each case, $P' =
\locv{n}{\tntin{m}.E}{B}{\vec{\sigma}}$ which is the same as $P$ and
thus well-typed.

\case{T-EnvOut}:

\noindent If the last rule in the derivation is T-EnvOut, then we know
from the form of the rule that $P$ must be the process term
$\locv{n}{\locv{m}{\locv{k}{\tntin{m}.E}{B_1}{\vec{\sigma}}}{B_2}{\vec{\rho}}}{B_3}{\vec{\gamma}}$
and $t$ must be the type $g : Group$.  From the semantics, there are
two subcases:

\subcase{Cap1}:

\noindent The $Cap1$ rule from the semantics provides the transition
$\ambop.E \derives{\ambop} E$, so $P'$ is
$\locv{n}{\locv{m}{\locv{k}{E}{B_1}{\vec{\sigma}}}{B_2}{\vec{\rho}}}{B_3}{\vec{\gamma}}$.
From the subderivations of the T-EnvOut typing rule, we know that $E : g
: Group$ so by multiple application of the T-Environ rule $P'$ is
well-typed.

\subcase{Cap2}:

\noindent From the $Cap2$ rule of the semantics, $\ambop.E$ has a
transition $\ambop.E \derives{\sigma} \ambop.E$ for each clock
$\sigma$ in $\timers$.  In each case, $P' =
\locv{n}{\locv{m}{\locv{k}{\tntin{m}.E}{B_1}{\vec{\sigma}}}{B_2}{\vec{\rho}}}{B_3}{\vec{\gamma}}$
which is the same as $P$ and thus well-typed.

\case{T-Open}:

\noindent If the last rule in the derivation is T-Open, then we know
from the form of the rule that $P$ must be the process term
$\locv{n}{\tntopen{m}.E \pc
  \locv{m}{F}{B_1}{\vec{\sigma}}}{B_2}{\vec{\rho}}$ and $t$ must be
the type $g : Group$.  For the case of $F$, we use the induction
hypothesis to retain well-typedness.  For $\tntopen{m}.E$, there are
three subcases in the semantics:

\subcase{Cap1}:

\noindent The $Cap1$ rule from the semantics provides the transition
$\ambop.E \derives{\ambop} E$, so $P'$ is $\locv{n}{E \pc
  \locv{m}{F}{B_1}{\vec{\sigma}}}{B_2}{\vec{\rho}}$.  From the
subderivations of the T-Open typing rule, we know that $E : g : Group$
so by application of the T-Par and T-Environ rules $P'$ is well-typed.

\subcase{Cap2}:

\noindent From the $Cap2$ rule of the semantics, $\ambop.E$ has a
transition $\ambop.E \derives{\sigma} \ambop.E$ for each clock
$\sigma$ in $\timers$.  In each case, $P' = \locv{n}{\tntopen{m}.E \pc
  \locv{m}{F}{B_1}{\vec{\sigma}}}{B_2}{\vec{\rho}}$ which is the same
as $P$ and thus well-typed.

\subcase{Open}:

\noindent The $Open$ rule from the semantics allows $P
\derives{\topen} P'$, so $P' = \locv{n}{E \pc F}{B_2}{\vec{\rho}}$.
From the subderivations of the T-Open typing rule, we know that $E : g
: Group$ and $F : g : Group$ so by application of the T-Par rule $P'$
is well-typed.

\case{T-ProcIn}:

\noindent If the last rule in the derivation is T-ProcIn, then we know
from the form of the rule that $P$ must be the process term $a.E \pc
\procin{a}{m}.F$ and $t$ must be the type $g \otimes g': Group$.
There are five subcases in the semantics, two of which we have already
proved for the case T-Act; for $a.E$, we know that it can progress by
the $Act$ or $Patient$ rule and still be typeable, and in these cases
$P'$ is also well-typed via the T-Par rule.

\subcase{Cap1}:

\noindent The $Cap1$ rule from the semantics provides the transition
$\ambop.E \derives{\ambop} E$, so $P'$ is $a.E \pc F$.  From the
subderivations of the T-ProcIn typing rule, we know that $F : g : Group$
so by application of the T-Par rule $P'$ is well-typed.

\subcase{Cap2}:

\noindent From the $Cap2$ rule of the semantics, $\ambop.E$ has a
transition $\ambop.E \derives{\sigma} \ambop.E$ for each clock
$\sigma$ in $\timers$.  In each case, $P' = a.E \pc \procin{a}{m}.F$
which is the same as $P$ and thus well-typed.

\subcase{ProcIn}:

\noindent The $ProcIn$ rule from the semantics allows $P
\derives{\tin} P'$, so $P' = \locv{m}{E}{B}{\vec{\sigma}} \pc F$.
From the subderivations of the T-ProcIn typing rule, we know that $E :
g : Group$ and $F : g : Group$ so by application of the T-Environ and
T-Par rules $P'$ is well-typed.

\case{T-ProcOut}:

\noindent If the last rule in the derivation is T-ProcOut, then we know
from the form of the rule that $P$ must be the process term
$\locv{n}{\locv{m}{a.E \pc
    \procout{a}{m}.F}{B_1}{\vec{\sigma}}}{B_2}{\vec{\rho}}$ and $t$
must be the type $g \otimes g': Group$.  There are five subcases in
the semantics, two of which we have already proved for the case T-Act;
for $a.E$, we know that it can progress by the $Act$ or $Patient$ rule
and still be typeable, and in these cases $P'$ is also well-typed via
the T-Par and T-Environ rules.

\subcase{Cap1}:

\noindent The $Cap1$ rule from the semantics provides the transition
$\ambop.E \derives{\ambop} E$, so $P'$ is $\locv{n}{\locv{m}{a.E \pc
    F}{B_1}{\vec{\sigma}}}{B_2}{\vec{\rho}}$.  From the subderivations
of the T-ProcIn typing rule, we know that $F : g : Group$ so by
application of the T-Par and T-Environ rules $P'$ is well-typed.

\subcase{Cap2}:

\noindent From the $Cap2$ rule of the semantics, $\ambop.E$ has a
transition $\ambop.E \derives{\sigma} \ambop.E$ for each clock
$\sigma$ in $\timers$.  In each case, $P' = \locv{n}{\locv{m}{a.E \pc
    \procout{a}{m}.F}{B_1}{\vec{\sigma}}}{B_2}{\vec{\rho}} = P$ and is
thus well-typed.

\subcase{ProcOut}:

\noindent The $ProcOut$ rule from the semantics allows $P
\derives{\tout} P'$, so $P' = \locv{n}{a.E \pc
  \locv{m}{F}{B_1}{\vec{\sigma}}}{B_2}{\vec{\rho}}$.  From the
subderivations of the T-ProcOut typing rule, we know that $E : g :
Group$ and $F : g : Group$ so by application of the T-Environ and
T-Par rules $P'$ is well-typed.

\end{proof}

