\chapter{Introduction}
\label{introduction}

\section{Rationale}

Recent changes in the direction of computer hardware development have
created an impasse in the domain of software engineering.  Over the
past few years, new microprocessors have not seen the same increase in
clock speed that has prevailed over previous decades.  Instead, the
use of multiple `cores' has become common, due largely to physical
limitations which prevent the individual elements of a single
processor core becoming any smaller.  As a result, the performance
benefits of these new processors arise not from being able to execute
a single task faster than before, but from the parallel execution of
many tasks.

However, this leads to a problem.  The existing dominant methods for
designing software systems are inherently sequential.  Current
imperative and object-oriented programming languages are still founded
on the principles of early computational models, such as the Turing
machine \cite{turing:36}.  These take a idealised view of events
whereby they always occur sequentially and in isolation.  Programs are
thus still effectively written as a sequence of reads and writes to a
form of memory.  The problem with this approach is that it runs into
major issues when the execution of other programs may cause changes to
memory outside the remit of the program.  Imagine Turing's model but
with multiple heads, each running separate programs yet still sharing
the same tape -- what happens if more than one head writes to the same
area of the tape?

\subsection{The Current Status Quo}

Concurrency is nothing new.  The concept of executing multiple programs
at once has been in use since multiprogramming was first introduced back
in the 1960s.  But the same underlying model has remained.  Parallelism
is still seen as an optimisation, beholden to the maintenance of the
sequential standard.  Utilising concurrency within a program remains
relegated to study as an advanced feature, seldom taught and even less
well practiced.  If parallelism is to become the dominant means of
exploiting the power of future hardware, this needs to change.

\section{Purpose of the Thesis}

\section{Contributions to Knowledge}

\section{Structure of the Thesis}

A standard feature of concurrent systems modelling is the use of
\emph{synchronisation} to model \emph{interaction}. Indeed, this is the
defining concept of CCS \cite{milner:ccs}, a process algebra commonly
used for modelling synchronous communication between two processes,
where one sends a signal and the other receives it at the same time (a
concept referred to as \emph{local synchronisation}).  However, it
cannot directly represent systems involving synchronisation of a sender
with an arbitrary number of recipient processes (known as \emph{global
synchronisation}) in a \emph{compositional} manner.  

For example, consider an environment inhabited by the members of a
factory team. The team receives a series of tasks to execute, which
require the team members to communicate with one another. At some point,
the first task is completed and the team moves on to the next. Modelling
this in a setting where the number of workers cannot be predicted
typically requires the use, either of \emph{global} synchronisation, or
else (which we reject) of an infinite set of defining equations.
Crucially, this means that the semantics of a broadcast agent cannot
suitably be represented using CCS.  If the agent is defined as
transmitting a signal to each of the recipients sequentially, through
multiple local synchronisations, then its semantics will become
non-compositional, because such behaviour depends upon the number of
recipients.  Each time a new recipient is introduced, or one of the
existing ones is removed, the semantics will have to be changed.

A solution to this deficiency lies in determining when all possible
synchronisations have taken place.  With this facility available, the
broadcast agent can recurse, transmitting signals, until this
condition holds. The family of abstract timed process calculi
(including TPL\cite{hennessy:tpl} and CaSE\cite{CaSE}) allow this by
extending CCS with \emph{abstract clocks}.  These don't represent real
time, with units such as minutes and seconds, but are instead used to
form synchronous cycles of internal actions followed by clock ticks.
A concept known as \emph{maximal progress} enforces the precedence of
internal actions over clock ticks, allowing the possible
synchronisations to be monitored.  When a synchronisation takes place,
it appears to the system as an internal action.  Thus, with maximal
progress, synchronisations prevent the clock from ticking, and as a
result, the occurrence of a clock tick also indicates that there are
no possible synchronisations.

However, the timed calculi mentioned above lack any notion of spatial
distribution or mobility. Thus, while they can adequately represent
large static systems, involving both local and global synchronisation,
they fail to model the ability of a system to change its topological
structure unilaterally. In contrast, the ambient calculus \cite{amb} and
its analogues are well positioned to model distributed systems (via
structures known as \emph{ambients}) in which resources can migrate from
one location to another. But, it suffers from similar deficiencies to
CCS when modelling global synchronisation.

This report presents the formal theory of the calculus of \emph{Typed
Nomadic Time} (TNT) \cite{hughes:nt}, which combines the abstract timed
calculus, CaSE, with notions of distribution and mobility from the
ambient calculus and its variants
(\cite{sangiorgi:mobsafeambients,controlledamb02}).  This allows the
creation of a compositional semantics for mobile component-based
systems, which utilise the notion of communication between arbitrary
numbers of processes within a mobile framework.  To extend the example
of a broadcast agent given above, this extension allow broadcasts to be
localised to a particular group of processes, which can change during
execution.  Current work on TNT is discussed in chapter
\ref{currentwork}, while chapter \ref{literaturereview} contains a
review of the existing literature in this area.  

Our long term aim is to leverage TNT as a foundation for creating
concurrent systems within a programmatic context. This will allow the
specification of system interactions to be shifted directly from the
theoretical domain into an implementation backed by a formal
methodology, helping in turn to improve industrial adoption of
concurrent techniques.  This is discussed along with future development
of the calculus in chapter \ref{futurework}.
