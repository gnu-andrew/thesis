\chapter{Introduction}
\label{introduction}

\section{Rationale}

Recent changes in the direction of computer hardware development have
created an impasse in the domain of software engineering.  Over the
past few years, new microprocessors have not seen the same increase in
clock speed that has prevailed over previous decades.  Instead, the
use of multiple `cores' has become common, due largely to physical
limitations which prevent the individual elements of a single
processor core becoming any smaller.  As a result, the performance
benefits of these new processors arise not from being able to execute
a single task faster than before, but from the parallel execution of
many such tasks.

However, this leads to a problem.  The existing dominant methods for
designing software systems are inherently sequential.  Current
imperative and object-oriented programming languages are still founded
on the principles of early computational models, such as the Turing
machine \cite{turing:36}.  These take a idealised view of events
whereby they always occur sequentially and in isolation.  Programs are
thus still effectively written as a sequence of reads and writes to a
form of memory.  The problem with this approach is that it runs into
major issues when the execution of other programs may cause changes to
memory outside the remit of the program.  Imagine Turing's model but
with multiple heads, each running separate programs yet still sharing
the same tape -- what happens if more than one head writes to the same
area of the tape?

In this thesis, we advocate a move towards systems where the focus is
on interaction between minimal sequential subsystems.  Rather than
building huge monolithic structures, the same result can be achieved
using a number of smaller components, running in parallel.  Such a
design has been suggested in varying forms over the years, but due to
the perceived future evolution of the microprocessor, this is now an
essential requirement, rather than a design ideal or optimisation.  We
also provide a formal grounding for such designs, based on academic
research which has been largely overlooked in the industrial sector.
Security also forms an inherent part of both the design and formal
model by allowing restrictions to be imposed on the communication
between individual components.

In the remainder of this chapter, we provide a brief overview of the
evolution of concurrent processing, highlighting current issues
arising from the flawed approach of maintaining a sequential design
which is becoming more and more distant from reality.  We also look at
how restricted mutability and an emphasis on intercommunication
between smaller, more specific processes can provide a better
solution, and how this approach has been adopted in the past with
varying results.  We close with a summary of the novelty of this work,
and an overview of how this will be covered in the later chapters of
this thesis.

\subsection{The Current Status Quo}

\subsubsection{Multiprogramming}

Concurrency is nothing new.  The concept of executing multiple
programs at once has been in use since \emph{multiprogramming} was
first introduced back in the 1960s.  But the same underlying model has
remained.  Parallelism is still seen as an optimisation, beholden to
the maintenance of the sequential standard.  Utilising concurrency
within a program remains relegated to study as an advanced feature,
seldom taught and even less well practiced.  If parallelism is to
become the dominant means of exploiting the power of future hardware,
this needs to change.

Multiprogramming was introduced as an efficiency measure.  At the
time, machines were available only on a per-institution rather than
per-user level, so a batch of \emph{jobs} were submitted to the
machine, each consisting of the program to run and any associated data
it needed to do so.  The machine ran a relatively simple
\emph{operating system}, which would take each job in turn and execute
a specified series of commands written in a batch job language.  Such
jobs would usually consist of reading in the program, compiling it if
necessary, and then executing it.  During execution, data was read in
and the results of computation output for the user to later digest.

It soon become clear that having an expensive processor sit idle while
input/output (I/O) operations took place was wasteful.  To solve this
problem, a new generation of machines were introduced which provided a
\emph{scheduler} as part of the operating system.  Instead of running
each job to completion before attempting the next, the system read in
multiple jobs to begin with, each forming a \emph{process} in memory.
These processes consisted not only of the code being executed, but
also included contextual information, such as the current instruction
being executed (the \emph{program counter}) and environmental data
(e.g. open file handles).

If a process being run by the system reached a point where it had to
wait for an I/O operation, the scheduler would move the process into a
\emph{blocked} state and perform a \emph{context switch} to begin the
execution of another.  Once the I/O operation was complete, the
blocked process would be reassigned to a \emph{ready} state, making it
again eligible for execution.

All this remained completely invisible to the running processes, each
of which appeared to be running in complete isolation.  The hardware
provided memory protection, which prevented a process from accessing
data outside its own memory space and they remained largely oblivious
to the fact that their execution was effectively being paused and then
resumed later.  The effect of such operation was only noticeable if
the running time of the process was recorded, as such results were now
dependent on factors such as system load and the arbitrary choices of
the scheduler.

Over time, schedulers have been extended so as to also switch when a
quantum of time allocated to a process has been depleted.  This
ensures a greater degree of fairness; a processor-intensive task which
rarely blocks can no longer become overly dominant.  This wasn't of
great importance for batch systems, as users submitted a job and then
collected the results later on.  In this context, just utilising the
time when a process was blocked had a significant impact on perceived
performance.  However, with a move towards first time sharing and then
personal computer systems, it became necessary to ensure that each
process was given time to execute on a regular basis, so the system
remained responsive.  This concept is referred to as
\emph{preemption}.

Finally, further performance enhancements were made possible by
allowing processes to have multiple threads of control and extending
the scheduler to enable switching between the individual threads
within a process.  The advantage of such threading is that the threads
share the same memory space and thus may interoperate more easily and
more efficiently.  The disadvantage is that it makes the possibility
of contention much more likely.

\subsubsection{Resource Contention}

Concurrency issues arise when a resource is accessed by multiple
processes or threads at the same time.  With threads, this is a
frequent occurrence as they run the same code and access the same
variables.  It also occurs with processes; although they have their
own memory space in which to operate, the resources provided by the
underlying operating system are shared by them all.  An obvious
example is the filesystem.  What happens if more than one process
tries to access a file at the same time?  Unless only reads occur, the
possibility of data corruption arises.

Such bugs, known as \emph{race conditions}, are difficult to reproduce
as they are heavily dependent on timing.  This is especially true of
single processor systems, where concurrency is merely simulated by the
scheduler switching between processes.  Whether or not file corruption
occurs depends on the choices made by the scheduler, which in turn
depends on a number of factors, such as system load.  If many
processes are competing for the processor, then there is less chance
of one which accesses the same file being picked.

A print spooler is a program which allows a printer (another shared
resource) to be used by many processes while maintaining separation
between individual jobs.  Without such a mechanism, one program may
write a few lines to the printer, and then be suspended by the
scheduler.  The program which is allowed to run next may then also
write to the printer, causing the user to end up with output from
different jobs mixed together.

Instead, the spooler tackles this concurrency problem by acting as an
mediator between the programs and the printer.  However, such an
application must be carefully designed to ensure it doesn't fall foul
to the same issue.  Imagine the spooler operates by reading a list of
files to print from a shared file.  When a process wants to add a new
job to the queue, it writes the filename as a new entry at the end of
the file:

\begin{verbatim}
int fd = open("/var/spool/print_jobs");
seek(fd, END_OF_FILE);
write(fd, "my_print_job");
close(fd);
\end{verbatim}

Problems arise because such an operation is \emph{non-atomic}; it is
possible that the process may be stopped by the scheduler while adding
a job to the list (e.g. after the \texttt{seek} function above), just
as it may be stopped while writing to the printer.  If this happens,
there is a possibility that whichever other process is scheduled in
its place could also choose to alter the queue.  The result of such a
collision depends on the timing:

\begin{enumerate}
\item If the first program only opened the file, or was just about to
  close it, then there will be no consequence.  In the first case, the
  first program will move to the end of the (now longer) file when it
  resumes and write its entry.  In the latter case, closing the file
  is just a matter of freeing resources and has no effect on the file
  itself.
\item If the first program seeked to the current end of the file, then
  on resumption, it will overwrite any data added in the meantime.  If
  the new data is longer than the older data, then the old data will
  simply be lost.  If it is shorter, the file will be corrupted.
\end{enumerate}

The solution to these sort of problems is to limit access to a
resource, so that a process is forced to wait its turn.

\subsubsection{Semaphores and Monitors}

Such access limitations can be imposed by a \emph{semaphore}, a
solution first proposed by Dijkstra\cite{semaphore}.  A semaphore
maintains an integer count which is manipulated by two operations:
\emph{up} and \emph{down}.  The count can be used to limit the number
of threads of control active in a particular region.  In effect, this
is akin to the scenario where a gate requires a token in order to
allow someone (a thread) to pass through, but the number of such
tokens is limited.  When a thread wants to pass through the gate, it
attempts to acquire a token by executing the down operation.  If the
count maintained by the semaphore is greater than zero, then it will
be decremented and the thread can proceed through the gate.  However,
if it is zero, there are no tokens left so the thread is forced to
wait until one of the existing tokens is returned.  Tokens are
returned by executing the up operation.

The up and down operations must be atomic; it should not be possible
for such an operation to be interrupted.  If they can, then the whole
purpose of the semaphore is defeated; a further solution would be
needed to resolve the possible concurrency issues that may occur
inside the semaphore itself.  Most operating systems provide such
atomicity by using support available at the processor level.

\emph{Binary semaphores}, where the count is either zero or one, are
very common.  Such semaphores can be implemented in a more simplified
form known as a \emph{mutex}, which maintains a binary state
(locked/unlocked) rather than a count.  Locking a mutex is equivalent
to decrementing the count to zero via a down operation, and unlocking
it is the same as performing an up to return its value to one.  The
usage pattern is the same for both: a thread first locks the mutex,
does its work and then unlocks the mutex to allow others access.

Mutexes can also be implemented at the file level as file locks,
providing a solution to the problem we encountered in the previous
section:

\begin{verbatim}
int fd = open("/var/spool/print_jobs");
flock(fd, LOCK_EX);
seek(fd, END_OF_FILE);
write(fd, "my_print_job");
flock(fd, LOCK_UN);
close(fd);
\end{verbatim}

The first call to \texttt{flock} acquires an exclusive lock
(\texttt{LOCK\_EX}) on the file referenced by \texttt{fd} (the file
descriptor returned by the operation which opens the file).  Let's
assume that this process is stopped by the scheduler after the
\texttt{seek} function executes and another process is allowed to run.
This second process executes the same program.  While it can
successfully acquire a file descriptor for the file through the
\texttt{open} function, the \texttt{flock} function will block trying
to obtain an exclusive lock.  This is because the lock is still held
by the original process which has been descheduled but has not yet
relinquished the lock.  When the original process is chosen again by
the scheduler, it can continue to write to the file, safe in the
knowledge that no other process has altered its contents in the
interim.  The final call to \texttt{flock} releases the lock so the
second process may now proceed.

Semaphores also have signalling capabilities; threads waiting to
perform a down operation are woken when an up occurs on the same
semaphore.  They can then retry the down operation again and return,
having decremented the value of the semaphore, should the operation
succeed on this attempt.  Given that there may be multiple waiting
threads, there is no guarantee that a thread will do so; for each up
operation, only one down operation will be successful and any other
threads will again be forced to wait.  Again, this race is why it is
essential that the down operation itself is atomic.

Suppose we want to implement a bounded buffer which is accessed by
multiple threads.  We need to use semaphores both to prevent possible
race conditions when modifications are made to the buffer, and to
stall threads when the buffer is full (in the case of adding a new
item) or empty (when retrieving an item).

As in our previous example, a binary semaphore or mutex can be used to
make modifications to the buffer appear atomic; a thread wanting to
operate on the buffer needs to first acquire the token and will be
unable to do so if another thread has already taken it.  Semaphores
can also be used to monitor the state of the buffer, and provide
notifications to the producer and consumer threads when the buffer
empties or fills up, respectively.

\begin{verbatim}
produce()
{
  item = produce_item();
  down(empty);
  down(mutex);
  add_item_to_buffer(item);
  up(mutex);
  up(full);
}

consume()
{
  down(full);
  down(mutex);
  item = remove_item_from_buffer();
  up(mutex);
  up(empty);
  consume_item(item);
}
\end{verbatim}

The above example provides an example implementation of such a buffer,
using three semaphores: \texttt{mutex}, \texttt{empty} and
\texttt{full}.  The \texttt{mutex} semaphore is a binary semaphore,
which ensures a thread has exclusive access to the buffer by making
modifications to the buffer appear atomic; although the thread can
still be interrupted, any other threads trying to execute
\texttt{down(mutex)} will be blocked until the original thread
relinquishes control.

The other semaphores are used to maintain a count of how many empty or
non-empty slots are available in the buffer.  As the buffer is filled,
the number of empty slots goes down and the number of non-empty slots
goes up.  The inverse is true when the buffer is emptied by the
\texttt{consume} function.  The \texttt{empty} mutex is initialised
with a value equal to the size of the buffer, while the \texttt{full}
mutex begins with a value of zero.

In the \texttt{produce} function, the thread first checks if there are
any empty slots by performing a \texttt{down} operation on the
\texttt{empty} mutex.  If the \texttt{empty} semaphore has a non-zero
value, as at the beginning, then there are available slots in the
buffer and the operation will return after decrementing the value by
one.  In this case, the thread can then proceed to lock the buffer
using the \texttt{mutex} and add an item to it.  It then releases the
\texttt{mutex} and performs an \texttt{up} operation on the
\texttt{full} semaphore, increasing the number of slots in use and
potentially allowing those threads waiting in the \texttt{consume}
function to proceed.  The \texttt{consume} function itself is
effectively the inverse of the \texttt{produce} function; it checks
the number of full slots to begin with, using the \texttt{full}
semaphore, and increases the number of empty slots when done.

The examples above are fairly simple, but already demonstrates some of
the inherent problems with the use of semaphores.  A successful
strategy for using them requires placing acquisition and release calls
in all affected locations and is extremely prone to error.  Suppose
one of the processes above never relinquishes the lock on the file.
Or a thread never performs an \texttt{up} on the mutex.  Other threads
or processes wishing to acquire the lock or mutex will be blocked
forever.  Similarly, it takes only one miscreant to access the shared
resource without attempting to acquire a lock to make the whole
process of locking redundant.

Semaphores don't scale well either.  For even a small program like the
buffer example above, three semaphores are required.  In such a
situation, the order of acquisition also becomes important.  If the
order is wrong or differs between code segments, deadlock can occur.
Deadlocks happen when each process or thread is waiting on a resource
held by another waiting process.  In the buffer example, simply
altering the order of the \texttt{down} calls in the \texttt{consume}
function is enough to create a potential deadlock situation.  If a
thread manages to acquire the mutex but is then forced to wait for an
up on the \texttt{full} semaphore, no other thread will be able to
acquire the mutex in the meantime.  Only in the unlikely situation
that a thread has been stopped between the \texttt{up(mutex)} and the
\texttt{up(full)} calls in the \texttt{produce} function would this
deadlock be resolved.  In most cases, the other threads will attempt
to acquire the mutex before reaching the required \texttt{up(full)}
call and so end up waiting forever.

By far the biggest issue with these kind of problems is
reproducability.  Just as with the race conditions they are trying to
avoid, bugs relating to semaphores may not always manifest themselves.
The example above is very likely to result in deadlock, as it just
requires the \texttt{consume} function to be called when the buffer is
empty and no other thread is accessing it.  Other issues can be much
harder to diagnose.

Take two processes, A and B, both of which are trying to acquire a
lock on the two files, \texttt{/etc/passwd} and \texttt{/etc/shadow}
in order to add a new user to the system.  If both processes acquire
the locks in the same order, then all is well.  If they don't, a
deadlock may occur.

Let's assume process A runs first.  It acquires a lock on
\texttt{/etc/passwd}.  At this point, A has used its allocated quantum
of CPU time and so is descheduled.  A context switch occurs and
process B begins to run.  If B begins by trying to acquire a lock on
\texttt{/etc/passwd}, then it will simply block as A already holds
this lock.  If, however, it tries to acquire a lock on
\texttt{/etc/shadow} first, this will succeed.  We then get stuck in a
deadlock; B blocks trying to acquire the lock on \texttt{/etc/passwd}
held by A, which will never be relinquished because A will be blocked
trying to acquire the lock on \texttt{/etc/shadow} held by B.  Such
problems occur simply through an ordering mismatch, but can be
extremely difficult to catch; in many situations, the process will
acquire both locks without being descheduled inbetween.

The solution to these problems is to abstract away from such intimate
details and allow the programmer to work at a more amenable level.
One attempt at doing so can be seen in the use of \emph{monitors}
\cite{mon1, mon2}.  Rather than worrying about the placement and
sequencing of individual acquisition and release calls, the programmer
simply denotes which sections of code must be run in mutual exclusion
from one another.  The compiler or virtual machine (depending on
whether the code is pre-compiled or not) then handles the process of
adding the required statements to ensure this.  The concept of
monitors is strongly linked to the idea of \emph{objects}, with the
same common idea of data encapsulation; all variables are private to
the object and inaccessible from the outside.  To read or modify the
data held by a monitor, one of its methods must be called.  Once a
thread is running code in a particular method, no other thread may
enter a method belonging to that monitor.  This ensures the thread
safety of the data without the issues of acquiring locks and lowers
the potential for deadlocks.

While this provides a better alternative to the use of binary
semaphores or mutexes, for a scenario such as the buffer example a
notification mechanism is required so that threads can wait for a
particular event to occur and be notified by other threads when it
does.  Monitors provide for this via the use of \emph{condition
  variables} and the \texttt{wait} and \texttt{signal} primitives.
Just as with semaphores, one thread calls the \texttt{wait} operation
on a particular condition variable and then another thread calls
\texttt{signal} on the same variable when the situation has changed.
The problem with this approach is that it is just as prone to error as
the use of semaphores; if the \texttt{wait} and \texttt{signal}
primitives are not used appropriately, then threads may be stalled.
It is still a very low-level solution.

Another issue with monitors, as implied above, is that they are
heavily reliant on support from the programming language being used.
While semaphores just require some means of performing an atomic
change to an area of memory, monitors need the compiler or virtual
machine to be intelligent enough to parse the monitor structures and
convert them into appropriate uses of more low-level locking
constructs.  One such language in which support is provided is Java,
as can be seen in the example below:

\begin{verbatim}
public class Buffer
{

  public static final int BUFFER_SIZE = 5;

  private int used = 0;
  private Object buffer[BUFFER_SIZE];
  
  public void produce()
  {
    Object item = produceItem();
    while (used == BUFFER_SIZE)
      wait();
    synchronized
    {
      buffer[used] = item;
      ++used;
    }
    notifyAll();
  }

  public void consume()
  {
    Object item;
    while (used == 0)
      wait();
    synchronized
    {
      --used;
      item = buffer[used];
    }
    notifyAll();
    consumeItem(item);
  }
}
\end{verbatim}

This is a implementation of the buffer example using monitors rather
than semaphores.  There are two main differences between the Java
implementation of monitors and that proposed in the academic
literature: the mutual exclusion is limited to blocks of code marked
with the \texttt{synchronized} keyword, rather than encompassing the
whole class, and the \texttt{wait} and \texttt{signal} operations are
realised as the \texttt{wait} and \texttt{notifyAll} methods of the
\texttt{Object} class rather than being functions applied to condition
variables.  One downside of these changes is that the addition of
selective mutual exclusion makes it prone to error; although it is
more efficient to not lock the entire class whenever any method is
called, this also means that one may forget to use the
\texttt{synchronized} keyword just as one may forget to perform the
appropriate operation on a semaphore.

The similarities and differences between monitors and semaphores can
be clearly seen by comparing the two buffer examples.  In the Java
version, the use of the \texttt{empty} and \texttt{full} semaphores is
replaced by a while loop and the use of \texttt{wait()} and
\texttt{notifyAll()}.  The value these depend on is also made explicit
in this version (see the variable \texttt{used}), whereas it is an
implicit part of the operations on the semaphores in the earlier
example.  When \texttt{produce} is called, it tests to see if the
buffer is full (the \texttt{used} count is equal to the size of the
buffer).  If it is, then \texttt{wait} is called.  The test takes
place in a \texttt{while} loop rather than a single \texttt{if}
statement so that the condition is tested again when the thread is
awoken by the \texttt{notifyAll()} call.  As before, if many threads
are waiting, it may be the case that the buffer is already full again
by the time a particular thread is allowed to execute.

The \texttt{synchronized} blocks are the equivalent of the the use of
the \texttt{mutex} semaphore; the opening bracket is the \texttt{down}
operation, while the closing bracket is the \texttt{up}.  Once a
thread is executing code inside one of these blocks, another thread
may not enter such a block, whether this be the same one or that in
the other method.  Modifications to the \texttt{buffer} and
\texttt{item} variables only take place within these blocks, thus
ensuring that only one thread can do so at a time.  Both variables are
marked \texttt{private}, making them invisible to code outside this
class.

What is clear from our comparison is that there are few advantages to
using monitors; they are prone to similar low level errors to those we
saw with semaphores, and they also require support from the language
being used, which may not always be possible.  Ideally, we instead
need to take a step back and limit the need for such locks altogether
by reducing the number of shared resources and the amount of
mutability inherent in our designs.  Not only are such designs prone
to error, but they also reduce the advantages of concurrent processing
(having to acquire a lock effectively makes operations single-threaded
once again) and are reliant on the existence of some form of shared
memory.  In distributed systems, shared resources do not exist
naturally but must instead be created artifically and may make
processing more inefficient.  In the future, we want to be able to
utilise the advantages of massively parallel systems and this can only
be achieved by reducing the need for resource contention.

\subsubsection{Interprocess Communication}

To achieve this, we need to focus on more short-lived processes which
interact directly with one another, rather than via the means of
shared resources.  This is nothing new.  However, it has never
achieved universal usage as a design paradigm because having to deal
with the kind of concurrency issues outlined above has been avoidable.
This is no longer the case.

Although mainstream development has migrated from procedural programs
to the \emph{object-oriented} paradigm, programs, once compiled, still
tend to be monolithic entities, with generally only a single thread of
control.  The notion of objects we see being used is not that of
Simula\cite{simula}, but one which is much more \emph{data-centric}.
These objects allow data to be separated out into neat little bundles
and stimulate reuse by allowing hierarchies of derived behaviour to be
created.  But there is no relationship between objects and threads;
when a method of an object is called, control switches from one object
to another.  If multiple threads are in use, then the objects are
shared between them and we see the kind of problems described above.

Solving this takes more than simply establishing a 1:1 relationship
between threads and objects.  For most designs, this would be terribly
inefficient and, in some cases, preposterous.  Because such designs
are centered around the data in use, we have objects such as a
\texttt{Borrower} in a library system.  We may have thousands of such
borrowers, many of which are inactive for weeks or months at a time.
Having a thread for each would simply be wasteful.  Instead, the
solution is to make objects \emph{task-centric}.

For a long time, UNIX systems have included the notion of pipelines
between processes.  For example, the command \texttt{du}, which
calculates disk usage, doesn't include an option to sort the results.
This is because there also exists a command, \texttt{sort} which can
order an arbitrary block of text in a number of ways.  As such, there
is no point adding duplicate functionality to \texttt{du} when its
output can just be fed in as input to \texttt{sort} for those who
desire this feature.

A pipeline is created in the shell by separating the two programs with
a \texttt{|} symbol.  For our example, \texttt{du -h | sort -n} would
do the job of outputting disk usage in human-readable form
(\texttt{-h}) and then sorting it numerically (\texttt{-n}).  A
similar solution can be applied programatically using system calls
such as \texttt{pipe}, \texttt{fork} and \texttt{execve}.  The pipe
allows the output of one program (\texttt{du}) to become the input of
another (\texttt{sort}).  Neither of the individual programs needs to
be aware that this is happening.  As far as \texttt{du} is concerned,
it is still sending output on its standard output channel.  The
difference is that this channel has been changed externally so as to
instead feed into a pipe, the other end of which forms \texttt{sort}'s
standard input channel.

This is a very simple solution, yet it elegantly solves the problem of
sharing the data between the two processes.  If a pipe was not used,
\texttt{du} would have to store its results somewhere for
\texttt{sort} to access.  This could then result in contention between
the two processes for access to the resource.  Instead, here the two
are working together rather than against each other by synchronising
the passage of data between them.  Each is independent of the other
and specific to its purpose.

A similar design tactic has been used in numerous areas.
\emph{Microkernels} such as Mach\cite{mach}, MINIX 3\cite{minix3} and
the GNU HURD\cite{hurd} also utilise this idea of synchronous
communication over a monolithic design based around shared resources.
In this context, it provides an essential stability and security
advantage; many services, such as drivers, file systems and network
protocols, can operate at the same level (or close to) as user-level
processes.

Some elements of the kernel must interact with the CPU at a very low
level.  This can only be achieved in the assembly language of the
processor and in a special \emph{protected} mode of operation.
However, it is not necessary for the entire kernel to function under
these conditions.  Device drivers are particularly notorious for
causing system instability by doing so.  Through running as part of a
monolithic kernel and assuming complete control, a buggy device driver
can cause the entire system to crash.

Instead, in MINIX 3, device drivers operate as separate privileged
processes.  Unlike normal user-level processes, they have the ability
to request direct access to hardware but such access is achieved
through message passing to the underlying core of the kernel.  The
majority of the driver's operation takes place in userspace and any
low-level access can be monitored and potentially prohibited.

The Mach and GNU HURD kernels (the former currently forming the basis
for the latter) take a similar approach.  The central mantra behind
this design is one of multiple servers, which provide different
functionality such as a file system or network service.  Apple also
adopted this design for XNU, the Mac OS X kernel, but, while basing it
on Mach, they largely reduced the design to a single server running a
monolithic BSD-based kernel.

The traditional objection against such designs has been performance.
Designs based on intercommunication have always tended to be more
elegant, but their usage has tended to be restricted to distributed
systems such as web services.  In these circumstances, any design
approach nesisitates utilising a potentially slow connection to
another system and having a central resource upon which all others
rely becomes disadvantageous, due to the potential for failure.

We believe it is time to reevaluate the benefits of systems focused on
intercommunication between specialised components.  With modern
systems, the potential performance disadvantage is becoming outweighed
by the benefits of a cleaner and more sustainable design.  With the
increasing prevalance of truly concurrent systems, more monolithic
systems will face a clear disadvantage, as the potential for
parallelism is severly reduced by contention for shared resources.

\section{Contributions to Knowledge}

Through this thesis, we present the following contributions to
knowledge which we believe to be novel:

\begin{enumerate}
\item The development of an algebraic process calculus (\ref{apc})
  with compositional global synchronisation (\ref{globsync}), mobility
  (\ref{mobility}) and security provision via the notion of `bouncers'
  -- see \ref{nt}.
\item The addition of a type system to the calculus which provides
  movement restriction via the group membership of processes -- see
  \ref{tnt}.
\item The realisation of the calculus as a design metholodogy through
  the implementation of its constructs as programmatic elements in the
  Java programming language -- see \ref{dynamite}.  This will allow
  the specification of system interactions to be shifted directly from
  the theoretical domain into an implementation backed by a formal
  methodology, helping in turn to improve industrial adoption of
  concurrent techniques.
\end{enumerate}

\section{Structure of the Thesis}

In the next chapter, we introduce existing research in to the area of
algrebriac process calculi through an exploration of the Calculus of
Concurrent Systems (CCS) \cite{milner:ccs}.  The following two
chapters focus on specific extensions to such calculi: global
synchronisation (\ref{globsync}) and mobility (\ref{mobility}).  In
chapter \ref{nt}, we introduce our own research in the form of the
Nomadic Time process calculus.  The following chapter (\ref{tnt})
extends this with a type system to create TNT (Typed Nomadic Time).
The second part of our research is covered in chapter \ref{dynamite}
with the development of the DynamiTE (Dynamic Theory Execution)
framework.  We close with suggestions for future work in chapter
\ref{futurework}.
