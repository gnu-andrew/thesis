\chapter{Introduction}
\label{introduction}

CCS \cite{milner:ccs} is commonly used for modelling synchronous
communication between two processes, where one sends a signal and the
other receives it at the same time (a concept referred to as
\emph{local synchronization}).  However, it cannot directly represent
systems involving synchronization of a sender with an arbitrary number
of recipient processes (known as \emph{global synchronization}) in a
\emph{compositional} manner.  Crucially, the semantics of a broadcast
agent cannot suitably be represented using CCS.  If the agent is
defined as transmitting a signal to each of the recipients
sequentially, through multiple local synchronizations, then its
semantics will become non-compositional, because such behaviour
depends upon the number of recipients.  Each time a new recipient is
introduced, or one of the existing ones is removed, the semantics will
have to be changed.

A solution to this deficiency lies in providing a way of determining
when all possible synchronizations have taken place.  With this
facility available, the broadcast agent can recurse, transmitting
signals, until this condition holds. The family of abstract timed
process calculi (including TPL\cite{hennessy:tpl} and CaSE\cite{CaSE})
allow this by extending CCS with \emph{abstract clocks}.  These don't
represent real time, with units such as minutes and seconds, but are
instead used to form synchronous cycles of internal actions followed
by clock ticks.  A concept known as \emph{maximal progress} enforces
the precedence of internal actions over clock ticks, allowing the
possible synchronizations to be monitored.  When a synchronization
takes place, it appears to the system as an internal action.  Thus,
with maximal progress, synchronizations prevent the clock from
ticking, and a result, the occurrence of a clock tick also indicates
that there are no possible synchronizations.

However, the timed calculi mentioned above lack any notion of
distribution or mobility. Thus, while they can adequately represent
large static systems, involving both local and global synchronization,
they fail to model more mobile systems, where the location of a
process can change during execution.  In contrast, the ambient
calculus \cite{amb} includes both distribution (via structures known
as \emph{ambients}) and mobility (by allowing these structures to be
moved, along with their constituent processes, during execution). But,
it suffers from similar deficiencies to CCS when modelling global
synchronization.

This report presents the calculus of \emph{Typed Nomadic Time} (TNT)
\cite{hughes:nt}, which combines the abstract timed calculus, CaSE, with
notions of distribution and mobility from the ambient calculus and its
safe variant (\cite{sangiorgi:mobsafeambients}).  This
allows the creation of a compositional semantics for mobile
component-based systems, which utilise the notion of communication
between arbitrary numbers of processes within a mobile framework.  To
extend the example of a broadcast agent given above, this extension
allow broadcasts to be localised to a particular group of processes,
which can change during execution.  Current work on TNT is discussed in
chapter \ref{currentwork}, while chapter \ref{literaturereview} contains
a review of the existing literature in this area.  Finally, chapter
\ref{futurework} discusses the future development of the calculus,
including possible case studies.
